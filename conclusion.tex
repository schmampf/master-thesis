\chapter{Conclusion}
% Mein Ziel in diser Arbeit war es die ferromagnetische resonance von einem supraleiter ferromagnet supraleiter (SFS) Contact aufzulösen. Dies ist auch gelungen für vernünftige Schichtdicken bis runter zu 3nm Co. Dies entspricht in etwa der Kohärenzlänge von Triplett-Cooperpaaren bei 300mK(Fußnote: Mit dem HelioxVL Aufbau, der auf Transport Messungen ausgelegt ist, lässt sich eine Grundtemperatur von 300mK erreichen) in Co. Das werte ich als einen wertvollen Zwischenschritt in dem fortlaufenden Project

% Jedoch habe ich nur sehr innerhalb der Ebene ausgedehnte Probengeometrien gewählt. Bei Proben, die auch für elektronische Transport messungen ausgelegt sind, muss diese Fläche um mehrere Größenordnungen runter scaliert werden. Nun ist daran das Problem, dass zum einen, die Signalstärke abnimmt. Die Stärke des gemessenen FMR Signals ist proportional zur Absorption im Ferromagneten, was sich leider auch proportional zum Probenvolumen verhält. Dadurch wird es bedeutend schwieriger bis unmöglich mit dem vector netzwerk analysator (VNA) die FMR aufzulösen. Zum anderen, wird sich durch die kleinere Probengeometrie, auch die Magnetisierung, das uniaxiale Anisotropie feld und damit die Resonanz frequenz verändern. Leider lassen sich diese Effekte nicht ohne weiteres vorhersagen. Damit muss für kleinere Probengeometrie auf alle Fälle wieder die FMR bestimmt werden, allerdings wird das gleichzeitig deutlich schwieriger, für kleiner Volumina. 

% Eine mögliche Idee ist es, nur die Fläche unter dem Innenleiter, nahe der Mitte zu nutzen. Dabei sollten Magnetfeldinhomogenitäten sowie das vermutlich dadurch induzierte Rauschen abnehmen. Um das Probenvolumen zu erhöhen, ohne die Probengeometrie zu verändern, könnte auch ein array aus Proben unter dem Innenleiter genutzt werden. 

% Nichtsdestotrotz konnten auch viele wertvolle Erkenntnisse über die verwendeten Aufbauten und Geräte. So konnte qualitativ gezeigt werden, dass der HeloixVL aufbau eine kleine Magnetisierung besitzt. Es konnten FMR signale, der gleichen Probe an beiden AUfbauten reproduziert werden. Ebenfalls konnten frequenz und magnetfeld sweep modus verglichen werden. Ich konnte sinnvolle Messparameter eingrenzen und eine klare Empfehlung für zukünftige Messungen aussprechen. Desweiteren habe ich die Kopplung zwischen Raum Temperatur und dem Rauschen der verwendeten Messelektronik untersucht. Ich habe ich die methode der frequenz domainen reflectrometrie erfolgreich implementieren können. Das hat sich als nützliches Werkzeug erwiesen, um abkühlbedingte Unterbrechungen der Messleitung zu lokalisieren. Desweiteren konnte ich eine temperaturabhängigkeit der Transmissions durch die Messleitung und Probe zeigen. Hier ist mit sicherheit interessant, die Messungen in einem niedrigeren Temperaturbereich zu wiederholen und feiner aufzulösen.

% Zusammenfassend werden meine wertvoll gewonnenen Erkenntnisse über die Ferromagnetische Resonanz eingang in die fortlaufende Forschung zu dynamischen ferromagnetischen Josephson Kontakten finden.

% My goal in this work was to resolve the ferromagnetic resonance (FMR) of a superconductor-ferromagnet-superconductor (SFS) contact. This has been achieved for reasonable film thicknesses down to $3\,$nm of Co. This roughly corresponds to the coherence length of triplet Cooper pairs at $300\,$mK \footnote{With the HelioxVL setup, which is designed for electronic transport measurements, a base temperature of $300\,$mK can be achieved.} in Co. I consider this a valuable intermediate step in the ongoing project

% However, I have chosen only very intra-plane extended sample geometries. For samples that are also designed for electronic transport measurements, this area must be scaled down by several orders of magnitude. Now the problem is that on the one hand, the signal strength decreases. The strength of the measured FMR signal is proportional to the absorption in the ferromagnet, which is unfortunately also proportional to the sample volume. This makes it significantly more difficult up to impossible to resolve the FMR with the vector network analyzer (VNA). On the other hand, due to the smaller sample geometry, the magnetization, the uniaxial anisotropy field and thus the resonance frequency will change. Unfortunately, these effects cannot be readily predicted. Thus, for smaller sample geometry, the FMR must be determined again in any case, but at the same time this becomes much more difficult for smaller volumes. 

% One possible idea is to use only the area under the inner conductor, close to the center. In this case, magnetic field inhomogeneities as well as the presumably induced noise should decrease. To increase the sample volume without changing the sample geometry, an array of samples under the inner conductor could also be used. 

% Nevertheless, it was also possible to gain many valuable insights into the setups and devices used. It could be shown qualitatively that the HeloixVL setup has a small magnetization. FMR signals of the same sample could be reproduced on both setups. Frequency and magnetic field sweep modes could also be compared. I was able to narrow down useful measurement parameters and make a clear recommendation for future measurements. Furthermore I investigated the coupling between room temperature and noise of the used measurement electronics. I was able to successfully implement the method of frequency domain reflectrometry. This has proven to be a useful tool to locate interruptions in the measurement line due to cooling. Furthermore, I was able to show a temperature dependence of the transmission through the measurement line and sample. Here it is certainly interesting to repeat the measurements in a lower temperature range and to resolve them more finely.

% In summary, my valuable findings on ferromagnetic resonance will find their way into ongoing research on dynamic ferromagnetic Josephson junctions.

The goal of my work was to investigate the ferromagnetic resonance (FMR) of a superconductor/ ferromagnet/superconductor (S/F/S) contact. This was achieved for reasonable film thicknesses down to $3\,$nm of Co. This thickness is comparable to the coherence length of singlet Cooper pairs in Co. Furthermore, I was able to obtain both the saturation magnetization and the single-axis anisotropy field of a $30$ and $3\,$nm thin Co layer. Likewise, I was able to obtain and explain volume-dependent components of the FMR.

However, only very-large-area sample geometries were investigated. For samples that are also designed for electronic transport measurements, this area must be reduced by several orders of magnitude. This leads to two challenges: First, the FMR signal strength scales with area, since the absorption in the ferromagnet is proportional to the sample volume and therefore to the signal strength. This makes it much more difficult, if not impossible, to resolve the FMR with the vector network analyzer (VNA). Second, the smaller sample geometry may change the magnetization, the uniaxial anisotropy field, and thus the FMR frequency, with no further prediction possible. Hence, for smaller sample geometries the FMR has to be redetermined, to exclude the effect of further geometrical constrains. To increase the volume and signal-to-noise ratio of a very small sample without changing its geometry, the possibility would be to use a whole series of the samples under the CPW.

In the future, only the area under the inner conductor, near the center, should be used. In this case, the alternating magnetic field would become more homogeneous and the signal-to-noise ratio of the FMR could increase. 

Furthermore, many valuable insights into the setups and devices used could also be gained. I was able to show qualitatively that the HelioxVL setup has magnetic components. FMR signals of the same sample could reproduced in both setups. I was also able to compare the frequency and magnetic field sweep modes. Likewise, I was able to narrow down useful measurement parameters and make a clear recommendation for future measurements. In addition, I investigated the coupling between laboratory temperature and noise of the measurement electronics used. I was able to successfully apply the method of frequency domain reflectometry. This proved to be a useful tool to locate interruptions in the measurement line due to thermal shrinking upon cooling. I was also able to show a temperature dependence of the transmission through the measurement line and the sample. It is of great interest to repeat the measurements in a lower temperature range and study in more detail.

In summary, my findings on ferromagnetic resonance will find their way into ongoing research on dynamic ferromagnetic Josephson junctions.
\newpage