%%%%%%%%%%%%%%%%%%%%%%%%%%%%%%%%%%%%%%%%%%%%%%%%%%%%%%%%%
%                                                       %
% Benutzerhandbuch für das Konstanz Beamer Theme        %
% Stand: 10.07.2015 - Michael Brendle (Version 0.3)     %
%                                                       %
%%%%%%%%%%%%%%%%%%%%%%%%%%%%%%%%%%%%%%%%%%%%%%%%%%%%%%%%%


% *WICHTIG*
% 1. Die Tex-Datei muss mittels XeLaTeX übersetzt werden, da somit
%    u.a. die Arial Schriftart eingebunden werden kann.
%    Bitte 2-mal übersetzten, damit auch das Inhaltsverzeichnis abgebildet wird.
% 2. Achten Sie darauf, dass alle notwendigen Pakete installiert sind.
%    Benötigte Pakete: xcolor, geometry, babel, fontenc, textpos, xunicode
%                      soul, ulem, ifthen, keycommand, tikz, calc, cmbright, fontspec 
%                      (siehe auch beamerthemeKonstanz.sty)
% 3. Bei Fragen, Hinweisen und Anregungen können Sie mich unter
%    michael.brendle@uni-konstanz.de kontaktieren.



% rgb ist hier dringend notwendig, damit die definierten Farben 
% des HSB Farbraums in den RGB Farbraum umgewandelt werden, da 
% einige Pakete keinen HSB Farbraum unterstuetzen.
\documentclass[rgb]{beamer}
\RequirePackage{etex}

\usepackage{layouts}

% Das neue Theme Konstanz verwenden
\usepackage{corporatedesign/beamerthemeKonstanz}
% Nummerierungstiefe: 3 ist hier fuer sub sub
\setcounter{secnumdepth}{3}


% Anpassung des Seitenverhältnis:
% Mögliche Eingabewerte: 
%   169 fuer (16:9)
%   43 fuer(4:3)
%   1610 fuer (16:10)
% 
%   * weitere werden noch folgen
%   * es wird noch an aspect Ratio des Beamer Themes angepasst
\format{169}


\begin{document}

% Initialisierung der Schrift
% Auf jeden Fall im Tex Dokument nach begin{document} belassen
\setmainfont{Arial}
\setsansfont{Arial}
\usebeamerfont{normalfont}


% Wichtig: Den Titel ebenfalls noch auf die maximal 4 Abschnitte einteilen
\title{Corporate Design: Benutzerhandbuch für das Beamer Paket}
\titleCorporateDesign{Corporate Design:}{Benutzerhandbuch für das}{Beamer Paket.}{}
\author{Oliver Irtenkauf} 
\date{10.07.2015}
\institute{Universität Konstanz}


\begin{frame}
	\titlepage
\end{frame}

\begin{frame}{figuresizes}

\begin{minipage}{.64\textwidth}
\begin{figure}
        \flushleft
        \includegraphics{analysis/Heliox_FMR_CoBulk30nm_upsweep/image_norm_cpd_beamer_23rd.pdf}
    \end{figure}
\end{minipage}
\begin{minipage}{.34\textwidth}
half page:
\begin{itemize}
    \item textwidth in inch: \printinunitsof{in}\prntlen{\textwidth}
    \item textwidth in mm: \printinunitsof{mm}\prntlen{\textwidth}
    
    \item textheight in inch: \printinunitsof{in}\prntlen{\textheight}
    \item textheight in mm: \printinunitsof{mm}\prntlen{\textheight}
\end{itemize}
\end{minipage}

\end{frame}
\begin{frame}{Frame Title}
    \begin{figure}
        \centering
        \includegraphics{analysis/Heliox_FMR_CoBulk30nm_upsweep/image_norm_cpd_beamer_full.pdf}
    \end{figure}
\end{frame}

\begin{frame}{Frame Title}
    \begin{figure}
        \centering
        \includegraphics{analysis/Heliox_FMR_CoBulk30nm_upsweep/image_cpd_beamer_half.pdf}
        \hspace{0.04\textwidth}
        \includegraphics{analysis/Heliox_FMR_CoBulk30nm_upsweep/rawimage_cpd_beamer_half.pdf}
    \end{figure}
\end{frame}

\begin{frame}[t]
	\frametitle{Inhaltsverzeichnis}
	\tableofcontents
\end{frame}


\section{Title, Section and Subsection Page}

\begin{frame}[t]
	\sectionpage
	\usebeamerfont{frametitle}%
		\textcolor{seeblau100}{Erzeugen der Titel Page}\\
	\usebeamerfont{normalfont}%
	\vskip 18pt
	\texttt{\textbackslash title\{Hier ganz normal den Titel eintragen\}}\\
	\texttt{\textbackslash titleCorporateDesign\{Hier den\}\{Titel in bis\}\{zu 4 Abschnitte\}\{aufteilen\}}\\
	\vskip 18pt
	\texttt{\textbackslash begin\{frame\} \textbackslash titlepage \textbackslash end\{frame\}}\\
	\vskip 18pt
	Es können zudem noch Formatierungsoptionen eingestellt werden. Mehr dazu im Abschnitt Markieren.
	\vskip 18pt
	\textit{Hinweis: Wenn der Titel in weniger als 4 Abschnitte aufgeteilt werden soll, dann die Argumente von vorne füllen und die Hinteren leer lassen.}
	\vskip 18pt
\end{frame}


\begin{frame}[t]
	\frametitle{Erzeugen der Section Page}
	\vskip 18pt
	\texttt{\textbackslash section\{Hier ganz normal den Titel eintragen\}}\\
	\vskip 18pt
	\texttt{\textbackslash begin\{frame\}}\\
	\texttt{\textbackslash sectionpage }\\
	\texttt{\textbackslash end\{frame\}}\\	\vskip 18pt
\end{frame}

\subsection{Subsection}
\begin{frame}[t]
	\subsectionpage
	Erzeugen der Subsection Page:
	\vskip 18pt
	\texttt{\textbackslash subsection\{Hier ganz normal den Titel eintragen\}}\\
	\vskip 18pt
	\texttt{\textbackslash begin\{frame\}}\\
	\texttt{\textbackslash subsectionpage }\\
	\texttt{\textbackslash end\{frame\}}\\
\end{frame}

\section{Frame}

\begin{frame}[t]
	\sectionpage
	Es kann natürlich direkt unter der \texttt{\textbackslash sectionpage} weitergeschrieben werden.
\end{frame}

\begin{frame}[t]
	\frametitle{Dies ist ein beliebiger Frame Title}
	\vskip 18pt
	\texttt{\textbackslash frametitle\{Dies ist ein beliebiger Frame Title\}}\\
	\vskip 18pt
	Ein \texttt{\textbackslash framesubtitle} wird aktuell noch nicht unterstützt.
\end{frame}


\section{Corporate Design Elemente}

\begin{frame}[t]
	\sectionpage
	Die vier Sättigungsstufen des \textbf{Seeblau} sind wie folgt definiert:\\
	\vskip 12pt
	\begin{beamercolorbox}[wd=0.5\textwidth, ht=3ex, dp=2ex, center]{seeblau100}
		\texttt{seeblau100}
	\end{beamercolorbox}
	\vskip 12pt
	\begin{beamercolorbox}[wd=0.5\textwidth, ht=3ex, dp=2ex, center]{seeblau65}
		\texttt{seeblau65}
	\end{beamercolorbox}
	\vskip 12pt
	\begin{beamercolorbox}[wd=0.5\textwidth, ht=3ex, dp=2ex, center]{seeblau35}
		\texttt{seeblau35}
	\end{beamercolorbox}
	\vskip 12pt
	\begin{beamercolorbox}[wd=0.5\textwidth, ht=3ex, dp=2ex, center]{seeblau20}
		\texttt{seeblau20}
	\end{beamercolorbox}
\end{frame}


\begin{frame}[t]
	Analog dazu die vier Sättigungsstufen der SW-Umsetzung:\\
	\vskip 12pt
	\begin{beamercolorbox}[wd=0.5\textwidth, ht=3ex, dp=2ex, center]{schwarz60}
		\texttt{schwarz60}
	\end{beamercolorbox}
	\vskip 12pt
	\begin{beamercolorbox}[wd=0.5\textwidth, ht=3ex, dp=2ex, center]{schwarz40}
		\texttt{schwarz40}
	\end{beamercolorbox}
	\vskip 12pt
	\begin{beamercolorbox}[wd=0.5\textwidth, ht=3ex, dp=2ex, center]{schwarz20}
		\texttt{schwarz20}
	\end{beamercolorbox}
	\vskip 12pt
	\begin{beamercolorbox}[wd=0.5\textwidth, ht=3ex, dp=2ex, center]{schwarz10}
		\texttt{schwarz10}
	\end{beamercolorbox}
\end{frame}


\begin{frame}[t]
	\frametitle{Schriftart}
	Als Schriftart wird die frei verfügbare Systemschrift \textbf{Arial} verwendet.\\
	\vskip 18pt
	Die Schriftgrößen können mittels des Makro\\
	\texttt{\textbackslash selectfontsize[baselinefaktor=<value>]\{<fontsize>\}}\\
    oder\\
    	\texttt{\textbackslash selectfontsize[baselinesize=<value>]\{<fontsize>\}}\\
	angepasst werden.\\[\baselineskip]
	Wird das optionale Argument weggelassen, wird automatisch der \texttt{baselinefaktor} 1.2 benutzt. Mehr dazu in der Tex und in der Style Datei.
\end{frame}

%%%%%%%%%%%%%%%%%%%%%%
% Schriftgröße       %
%%%%%%%%%%%%%%%%%%%%%%

% Da innerhalb eines Plakates es meistens mehrere verschiedene Schriftgrößen
% gibt, steht hier das Makro
%
%     \selectfontsize
%
% zur Verfügung.
%
% Dieses Makro besitzt ein unbedingt notwendiges Argument und ein optionales
% Feld, indem Key-Value Pairs übergeben werden können.
%
%     \selectfontsize[<Key Value Pairs>]{<Schriftgröße>}
%
% Das Argument hat dabei folgende Bedeutung:
% 
%   1. Argument:         Hier wird die neue Schriftgröße angegeben, die verwendet werden
%                        soll.
% Die weiteren Formatierungsoptionen werden alle innerhalb des optionalen Argumentes mittels
% Key-Value Pairs bestimmt.
%
% Dabei stehen folgende Optionen zur Verfügung:
%
%     baselineskip    Hier wird der baselineskip angegeben, welcher verwendet werden soll
%                     Mögliche Werte:
%                         0      Ist dieser 0, dann wird der baselinefaktor verwendet  
%                         sonst
%                    Standardwert: 0
%
%     baselinefaktor Hier wird der Faktor angegeben, der verwendet wird, um den neuen
%                    baselineskip zu berechnen.
%                    Dieser wird nur benutzt, falls der baselineskip 0 beträgt.
%                    
%                        baselineskip = baselinefaktor * #1
%
%                    Standardwert: 12/10
%
%                    Da hier keine Fließkommazahl in der Dezimalschreibweise angegeben werden
%                    kann, müssen diese als Brüche repräsentiert werden, wie z.b. 12/10 anstatt
%                    1.2.

\begin{frame}[t]
	\frametitle{Verschiedene Schriftgrößen}
	Die standardmäßig eingestellten Schriftgrößen können auch in der \texttt{beamerthemeKonstanz.sty} verändert werden.\\[\baselineskip]
	\selectfontsize{12pt} Ich bin eine sehr kleine Schriftgröße\\
	\selectfontsize{20pt} Ich bin schon größer\\
	\selectfontsize{28pt} Ich bin noch größer\\
	\selectfontsize{36pt} Ich bin ziemlich groß\\
	\selectfontsize{44pt} Ich bin rießig\\
\end{frame}

\subsection{Markieren}

%%%%%%%%%%%%%%%%%%%%%%%%%
% CD Element: Markieren %
%%%%%%%%%%%%%%%%%%%%%%%%%

% Um einen Text mit Hilfe des Markieren Elements des Corporate Design hervorzuheben,
% steht das Makro
%
%     \markieren
%
% zur Verfügung.
%
% Dieses Makro besitzt vier unbedingt notwendige Argumente und ein optionales
% Feld, indem weitere EIgenschaften festgelegt werden könnnen..
%
%     \markieren[Optionen per Key-Value Pair]{<Zeile 1>}{<Zeile 2>}{<Zeile 3>}{<Zeile 4>}
%
% Die Argumente haben dabei folgende Bedeutung:
%
%   1. - 4. Argument:    Hier werden nun die eigentlichen Zeilen übergeben.
%
%                        Wichtig dabei ist es, dass die Aufteilung der Zeilen manuell erfolgen muss durch
%                        die Argumente, da nur somit sichergestellt werden kann, dass bspw. Treppeneffekte
%                        nicht auftreten und somit der Benutzer alle Freiheiten bei der Aufteilung besitzt.
%
%                        Sollten nicht alle Zeilen verwendet werden, dann müssen die hinteren Brackets
%                        leer gelassen werden, wie beispielsweise bei der Headline
%
% Die wichtigen Formatierungsoptionen werden alle innerhalb des optionalen Argumentes mittels
% Key-Value Pairs bestimmt.
%
% Dabei stehen folgende Optionen zur Verfügung:
%
%   align                Hier kann angegeben werden, ob das komplette Objekt
%                        links- oder rechtsbündig angeordent werden soll.
%                      
%                        Der Standardwert ist "left" und somit linksbündig.
%
%                        Für eine rechtsbündige Anordnung muss hier der Wert "right" hinterlegt werden.
%
%   vertical             Hier wird angegeben, ob der Inhalt der Zeilen zentriert werden soll oder
%                        überall an der gleichen Baseline ausgerichtet werden soll.
%                       
%                        Dies kann mittels der Wörter "center" und "base" eingestellt werden.
%                        Dabei ist "center" als Standardwert festgelegt.
%
%                        Der Unterschied besteht darin, dass bei Zeilen die Buchstaben mit einer Tiefe
%                        enthalten, wie g, p oder q, anders zentriert werden als welche ohne Buchstaben
%                        mit einer Tiefe.
%
%                        Da dies ein wenig Geschmackssache ist, werden hier beide Varianten zur Verfügung
%                        gestellt, wobei "center" primär verwendet werden soll, und "base" eher wenn
%                        Buchstaben mit einer Tiefe in den Zeilen enthalten sind.

\begin{frame}[t]
	\subsectionpage
	Das \textbf{Markieren-Element} kann neben der \texttt{\textbackslash titlepage} problemlos mit\\
	\hskip 1cm \texttt{\textbackslash markieren[<Optionen>]\{<Zeile 1>\}\{<Zeile 2>\}\{<Zeile 3>\}\{<Zeile 4>\}}\\
	eingesetzt werden.\\[\baselineskip]
	Zudem stehen noch zwei optionale Argumente \texttt{align} und \texttt{vertical} zur Verfügung. Mittels \texttt{align} und den Werten \texttt{left} bzw. \texttt{right} kann die Ausrichtung des Markieren-Objektes festgelegt werden. Mit der Option \texttt{vertical} und den Werten \texttt{center} und \texttt{base} kann die Ausrichtung innerhalb der Zeilen festgelegt werden.
	\vskip 18pt
	\markieren{Ich bin}{eine}{Headline}{}\\

\end{frame}

\begin{frame}[t]
	\frametitle{Hier sind noch mehr Markieren-Elemente}%
	\vskip 18pt%
	\selectfontsize{28pt}%
	\begin{flushright}%
	\markieren[align=right]{\textbf{Ich bin}}{\textbf{eine}}{\textbf{Headline}}{}%
	\end{flushright}%
	\markieren[vertical=base]{\textbf{Erste Zeile}}{\textbf{von einer}}{\textbf{vierzeiligen}}{\textbf{Headline}}%
\end{frame}





\subsection{Unterstreichen}

%%%%%%%%%%%%%%%%&%%%%%%%%%%%%%%
% CD Element: Unterstreichung %
%%%%%%%%%%%%%%%%%&%%%%%%%%%%%%%

% Um einen Text mit Hilfe des Unterstreichen Elements des Corporate Design hervorzuheben,
% steht das bereits bekannte Makro
%
%     \underline
%
% zur Verfügung, welches an die Anforderungen des Corporate Designs angepasst wurde.
%
% Dieses Makro besitzt ein notwendiges Argument
%
%     \underline{1. Argument}
%
% Das Argument hat folgende Bedeutung:
% 
%   1. Argument: Hier wird der zu unterstreichende Text hinterlegt.
%
% Wichtig ist noch zu wissen, dass auch Textbrüche ohne Probleme durchgeführt werden können.
%
% Zudem können weitere Formatierungen, wie bold oder italic innerhalb des Argumentes angewendet
% werden.
%
% Die Dicke der unterstrichenen Linie passt sich dabei der aktuell verwendeten Textgröße an.

\begin{frame}[t]
	\subsectionpage
	Das \textbf{Unterstreichen-Element} wird wie gewohnt mittels\\
	\hskip 1cm \texttt{\textbackslash underline\{<text>\}}\\
	eingesetzt. Das Makro wurde dafür entsprechend angepasst. Möchte man einen fetten unterstrichenen Text haben kann der zu unterstreichende Text einfach mittels \texttt{\textbackslash textbf\{\ldots\}} ergänzt werden: \texttt{\textbackslash underline\{\textbackslash textbf\{Ich bin der Anfang von einem Fließtext mit Unterstreichen\}\}} \\
	\vskip 18pt
	\underline{\textbf{Ich bin eine Subline mit Unterstreichen}}\\
	\vskip 18pt
	\selectfontsize[baselinefaktor=16/10]{16pt}\underline{\textbf{Ich bin der Anfang von einem Fließtext mit Unterstreichen}} und ich bin der weiterführende Fließtext\ldots\\
	\end{frame}


\subsection{Merken}

%%%%%%%%%%%%%%%%%%%%%%
% CD Element: Merken %
%%%%%%%%%%%%%%%%%%%%%%

% Um einen Text mit Hilfe des Merken Elements des Corporate Design hervorzuheben,
% steht das Makro
%
%     \merken
%
% zur Verfügung.
%
% Dieses Makro besitzt drei unbedingt notwendige Argumente
%
%     \merken{1. Argument}{2. Argument}{3. Argument}
%
% Die Argumente haben dabei folgende Bedeutung:
% 
%   1. Argument: Hier wird die Breite des kompletten Objektes angegeben. Da das
%                Merken Objekt quadratisch ist, wird hier sowohl die Breite als auch
%                die Höhe angegeben.
%
%   2. Argument: Hier wird die Subline des Merken Elementes angegeben, die direkt unter der
%                Zeile mit dem X folgt (siehe auch Corporate Design Manual).
%
%   3. Argument: Hier wird der eigentliche Inhalt angegeben. Wichtig hierbei ist es,
%                dass dieser Inhalt an die untere Kante des Merken Elementes orientiert ist.
%                Somit entgegen der Subline (2. Argument), welche an die obere Kante abzüglich
%                der Zeile mit dem X orientiert ist.
%
% Hier folgt noch eine grafische Darstellung der Argumente:
%
%    |<------- 1. Argument ------->|    
%
%    -------------------------------    -
%    |                           X |    ^
%    | Subline (2. Argument)       |    |
%    |                             |    1
%    |                             |    .
%    |                             |    A
%    |                             |    r
%    |                             |    g
%    |                             |    u
%    |                             |    m
%    |                             |    e
%    |                             |    n
%    |                             |    t
%    |                             |    |
%    | Inhalt (3. Argument)        |    v
%    -------------------------------    -
%
% Wichtig ist noch zu wissen, dass die Linienstärke und die Größe des X in der rechten oberen Ecke an die
% Höhe / Breite des Merken Elements dynamisch angepasst ist. 


\begin{frame}[t]
	\subsectionpage
	Das \textbf{Merken-Element} wird mit dem Makro\\
	\hskip 1cm \texttt{\textbackslash merken\{<Breite/Höhe>\}\{<Subline>\}\{<Inhalt>\}}\\
	eingesetzt. Da das Merken-Element quadratisch ist, muss nur eine Größe angegeben werden. Ist keine Subline oder Inhalt erwünscht, sollen die dafür vorgesehenen Brackets einfach leer gelassen werden.\\[\baselineskip]
	\selectfontsize{12pt}\merken{5cm}{}{\textbf{Hier steht etwas,} dass ich mir unbedingt merken oder das ich gesondert hervorheben möchte.}
\end{frame}

\begin{frame}[t]
	\frametitle{Hier sind noch mehr Beispiele}%
	\vskip 18pt%
	\selectfontsize{28pt}%
	\merken{7.5cm}{\selectfontsize{12pt}Bachelorstudiengang}{\textbf{Politik- und\\Verwaltungs-\\wissenschaft}}\selectfontsize{10pt}%
	\hskip 1cm\merken{6cm}{}{\textbf{Kontakt}\\[0.5\baselineskip]Prof. Dr. Guido Burkard\\Universität Konstanz\\Fachbereich Physik\\Universitätsstraße 10\\78464 Konstanz\\+49 7531 88-5256\\guido.burkard@uni-konstanz.de\\[0.5\baselineskip]\textbf{– uni-konstanz.de}}%
	\hskip 1cm\merken{4.5cm}{}{Die Universität Konstanz ist seit 2007 in allen drei Förderlinien der Exzellenzinitiative erfolgreich.}%
\end{frame}

\subsection{Block}

%%%%%%%%%%%%%%%%%%%%%%
% CD Element: Block  %
%%%%%%%%%%%%%%%%%%%%%%

% Mit dem Makro
%
%     \cdblock[Optionen per Key-Value Pair]{<Headline>}{<Spalte 1>}{<Spalte 2>}{<Spalte 3>}{<Spalte 4>}{<Spalte 5>}{<Spalte 6>}{<Spalte 7>}{<Spalte 8>}
%
% können Block-Elemente für z.B. wisschenschaftliche Inhalte erstellt werden.
%
% Dieses Makro besitzt 9 erforderliche Elemente, die bei jedem Aufruf angegeben werden müssen. Dabei 
% ist es natürlich mögliche Argumente leer zu lassen, falls man diese nicht benötigt. Dies hat jedoch
% keinen Einfluss auf die Anzahl an Spalten. Diese müssen separat im Optionenargument angegeben werden
% mittels des Schlüssels columnnum (siehe weiter unten).
%
% Die Argumente haben dabei folgende Bedeutung:
%
%   1. Argument: Inhalt der Headline
%   2. Argument: Inhalt der 1. Spalte
%   3. Argument: Inhalt der 2. Spalte
%   4. Argument: Inhalt der 3. Spalte
%   5. Argument: Inhalt der 4. Spalte
%   6. Argument: Inhalt der 5. Spalte
%   7. Argument: Inhalt der 6. Spalte
%   8. Argument: Inhalt der 7. Spalte
%   9. Argument: Inhalt der 8. Spalte
%
%
% Die wichtigen Formatierungsoptionen werden diesmal alle innerhalb des optionalen Argumentes mittels
% Key-Value Pairs bestimmt.
%
% Dabei stehen folgende Optionen zur Verfügung:
%
%     thick          Hier wird die Dicke der Linie bestimmt.
%                    Die Pfeile werden generell mit der doppelten Dicke gezeichnet!
%                    Standardwert: \boxlinewidth
%
%     color          Hier wird die Farbe der Linie angegeben
%                    Es sollten nur die folgenden Farben benutzt werden:
%                        seeblau100
%                        seeblau65
%                        seeblau35
%                        seeblau20
%                        black
%                        schwarz60
%                        schwarz40
%                        schwarz20
%                        schwarz10
%                    Standardwert: seeblau100
%
%     width          Hier wird die Breite des Blocks angegeben
%                    Standardwert: \paperwidth
%
%     columnnum      Hier werden die Anzahl an Spalten definiert
%                    Standardwert: 4
%
%     headlinesep    Hier wird der Abstand zwischen der Headline und den Spalten angegeben
%                    Standardwert: Aktuelle Schriftgröße
%
%     columnspace    Hier wird der Abstand zwischen den Spalten angegeben
%                    Standardwert: Doppelte Schriftgröße
%
%     block          Hier kann angegeben werden, ob man einen Rahmen um diesen Block haben möchte
%                    Mögliche Werte: true, false
%                    Standardwert: false
%
%     inner          Hier kann angegeben werden, ob zwischen den Spalten Trennlinien haben möchte
%                    Mögliche Werte:
%                        false    keine Trennlinien
%                        short    Trennlinien, die so lange sind, wie der längste Nachbar (entweder der
%                                 linke oder rechte Nachbar
%                        long     Trennlinien, die bis nach ganz unten gehen. Sie sind also so lang
%                                 wie die längste Spalte
%                    Standardwert: false
%
%     inner1,        Hier kann für jeden Zwischenraum der Spalte exakt angegeben werden, ob Trennlinien
%     inner2,        existieren sollen und falls ja, wie lang sie sein sollen. Diese Werte werden jedoch
%     inner3,        nur berücksichtigt, wenn inner=false ist. Ansonsten ist inner stärker.
%     inner4,        Mögliche Werte:
%     inner5,           false    keine Trennlinien
%     inner6,           short    Trennlinien, die so lange sind, wie der längste Nachbar (entweder der
%     inner7,                    linke oder rechte Nachbar
%                       long     Trennlinien, die bis nach ganz unten gehen. Sie sind also so lang
%                                wie die längste Spalte
%                    Standardwert: false
%
%     outerleft,     Hier kann angegeben werden, ob links (rechts) der ersten Spalte eine Trennlinie existieren soll.
%     outerright     Mögliche Werte
%                        false    keine Trennlinien
%                        short    Trennlinien, die so lange sind, wie der direkte Nachbar (bei outerleft die 1. Spalte
%                                 und bei outerright die letzte Spalte
%                        long     Trennlinien, die bis nach ganz unten gehen. Sie sind also so lang
%                                 wie die längste Spalte
%                        verylong Trennlinie geht von oben nach unten, sowie ein halber columnspace nach innen.
%                    Standardwert: false
%
%     outertop,      Hier kann angegeben werden, ob oberhalb (unterhalb) des Blocks eine Trennlinie, oder ein Pfeil existieren soll.
%     outerbottom    Mögliche Werte
%                        false    keine Trennlinien
%                        long     Trennlinien, die von links nach rechts geht
%                        verylong Trennlinie, die von links nach rechts geht, sowie ein halber columnspace nach oben (unten).
%                        arrow    Pfeil, der aus der Trenlinie verylong besteht und in der Mitte einen Pfeil nach oben (unten)
%                                 besitzt.
%                    Standardwert: false
%
%     arrowtop1left, Hier kann angegeben werden, ob zwei Spalten oberhalb des Blocks mittels eines Doppelpfeils verbunden werden
%     arrowtop1right sollen. Da es maximal 8 Spalten sind, können auch nur maximal 4 Paare bestimmt werden.
%     arrowtop2left  Ein Paar besteht somit aus einer linken und einer rechten Spalte.
%     arrowtop2right Mögliche Werte:
%     arrowtop3left      0   Keine Auswahl
%     arrowtop3right     1-8 Auswahl einer Spalte von 1 bis 8
%     arrowtop4left  Standardwert: 0
%     arrowtop4right Sollte der linke Wert nicht kleiner als der rechte Wert sein, so werden keine Pfeile gezeichnet. Das gleiche
%                    gilt für Werte, die außerhalb des Bereichs liegen.
% 
%
% Wichtig ist noch zu wissen, wie die Breite letztendlich berechnet wird:
% Da es auch links und rechts der Spalten Trennlinien oder Pfeile geben kann, ist links der 1. und rechts der
% letzten Spalte ebenfalls ein columnspace vorgesehen.
% Somit wird die Spaltenbreite wie folgt berechnet
%
%     blockcolumnwidth = (width - (columnspace * (columnnum + 1))) / columnnum
%


\begin{frame}[t]
\subsectionpage%
Das \textbf{Block-Element} welches vor allem auf Plakaten vorkommt, kann mit dem Makro\\%
\hskip 1cm \texttt{\textbackslash cdblock[<Optionen>]\{<Headline>\}\{<Spalte 1>\}\ldots\{<Spalte 8>\}}\\%
eingesetzt werden. Die ganzen Optionen können in der Tex und in der Style Datei nachgelesen werden.\\[\baselineskip]%
\selectfontsize{14pt}\cdblock[width=\textwidth-1cm, columnnum=4, columnspace=1cm, outerbottom=arrow, outertop=verylong, inner=long]{}{\textcolor{seeblau100}{\textbf{Spalte 1}}\\[\baselineskip]Dies ist die erste Spalte und hier kann eigentlich so alles mögliche stehen.}{\textcolor{seeblau100}{\textbf{Spalte 2}}\\[\baselineskip]Dies ist die zweite Spalte und hier kann auch wieder eine ganze Menge stehen}{\textcolor{seeblau100}{\textbf{Spalte 3}}\\[\baselineskip]Auch hier kann das ein oder andere wichtige stehen, oder sogar sehr wichtiges}{\textcolor{seeblau100}{\textbf{Spalte 4}}\\[\baselineskip]Und auch in der vierten Spalte findet man sicherlich einiges}{}{}{}{}%
\end{frame}

\begin{frame}[t]
\frametitle{Weitere Beispiele}
\vskip 18pt
\selectfontsize{14pt}\cdblock[width=\textwidth-1cm, columnnum=1, columnspace=1cm, outerleft=verylong, outerright=verylong]{\selectfontsize{18pt}\textcolor{seeblau100}{\textbf{Beweis}}}{Dieses Block-Element kann natürlich für Beweise und Beispiele sehr gut eingesetzt werden. Dabei sollte es vor allem dem Block-Element des Beamer Paketes vorgezogen werden.}{}{}{}{}{}{}{}\\[2\baselineskip]%
\cdblock[width=\textwidth-1cm, columnnum=1, columnspace=1cm, block=true]{\selectfontsize{18pt}\textcolor{seeblau100}{\textbf{Beispiel}}}{Dieses Beispiel hat einen kompletten Rahmen}{}{}{}{}{}{}{}%
\end{frame}

\subsection{Linien und Pfeile}

%%%%%%%%%%%%%%%%%%%%%%%%%%%%%%%%%%%%%%%%%%%%%%
% CD Element: Linie (mit optionalen Pfeilen) %
%%%%%%%%%%%%%%%%%%%%%%%%%%%%%%%%%%%%%%%%%%%%%%

% Mit dem Makro
%
%     \cdline[Optionen per Key-Value Pair]{<Länge der Linie>}
%
% kann eine Linie mit einer bestimmten Länge erstellt werden.
%
% Dieses Makro besitzt 1 erforderliches Element, welches bei jedem Aufruf mit angegeben
% werden muss.
%
% Das Argument hat dabei folgende Bedeutung.
%
%   1. Argument: Länge der erzeugten Linie
%
%
% Die wichtigen Formatierungsoptionen werden alle innerhalb des optionalen Argumentes mittels
% Key-Value Pairs bestimmt.
%
% Dabei stehen folgende Optionen zur Verfügung:
%
%     thick          Hier wird die Dicke der Linie bestimmt
%                    Standardwert: \boxlinewidth
%
%     mode           Hier wird angegeben, ob die Linie horizontal oder vertikal ausgerichtet werden soll
%                    Mögliche Werte
%                        horizontal
%                        vertical
%                    Standardwert: horizontal
%
%     color          Hier wird die Farbe der Linie angegeben
%                    Es sollten nur die folgenden Farben benutzt werden:
%                        seeblau100
%                        seeblau65
%                        seeblau35
%                        seeblau20
%                        black
%                        schwarz60
%                        schwarz40
%                        schwarz20
%                        schwarz10
%                    Standardwert: seeblau100
%
%     arrowleft      Hier wird angegeben, die Linie am linken (vertical: oberen) Ende mit einem Pfeil enden soll
%                    Mögliche Werte:
%                        true
%                        false
%                    Standardwert: false
%
%     arrowright     Hier wird angegeben, die Linie am rechten (vertical: unteren) Ende mit einem Pfeil enden soll
%                    Mögliche Werte:
%                        true
%                        false
%                    Standardwert: false


\begin{frame}[t]
\subsectionpage%
Die \textbf{Linien-Elemente} welche vor allem auf Plakaten vorkommen, können mit dem Makro\\%
\hskip 1cm \texttt{\textbackslash cdline[<Optionen>]\{<Länge>\}}\\%
eingesetzt werden. Mit den Optionen, welche Sie in der Tex und Style Datei finden, kann die Linienstärke, Ausrichtung, Farbe und die Pfeile angepasst werden.\\[\baselineskip]%
\cdline[thick=8pt, arrowleft=true, arrowright=true]{10cm}\\
\cdline[color=seeblau20, thick=8pt, mode=vertical]{1cm} \hskip 1cm \cdline[color=seeblau35, thick=8pt, mode=vertical]{1.5cm} \hskip 1cm \cdline[color=seeblau65, thick=8pt, mode=vertical]{2cm} \hskip 1cm \cdline[thick=8pt, mode=vertical]{2.5cm} \hskip 1cm \cdline[color=seeblau65, thick=8pt, mode=vertical]{2cm} \hskip 1cm \cdline[color=seeblau35, thick=8pt, mode=vertical]{1.5cm} \hskip 1cm \cdline[color=seeblau20, thick=8pt, mode=vertical]{1cm} \hskip 1cm \cdline[color=schwarz10, thick=8pt, mode=vertical]{1cm} \hskip 1cm \cdline[color=schwarz20, thick=8pt, mode=vertical]{1.5cm} \hskip 1cm \cdline[color=schwarz40, thick=8pt, mode=vertical]{2cm} \hskip 1cm \cdline[color=schwarz60, thick=8pt, mode=vertical]{2.5cm} \hskip 1cm \cdline[color=schwarz40, thick=8pt, mode=vertical]{2cm} \hskip 1cm \cdline[color=schwarz20, thick=8pt, mode=vertical]{1.5cm} \hskip 1cm \cdline[color=schwarz10, thick=8pt, mode=vertical]{1cm}
\end{frame}

\subsection{Klammern}

%%%%%%%%%%%%%%%%%%%%%%%%%%%%%%%%%%%%%%%%%%%%%%%%
% CD Element: Klammer (mit optionalen Pfeilen) %
%%%%%%%%%%%%%%%%%%%%%%%%%%%%%%%%%%%%%%%%%%%%%%%%

% Mit dem Makro
%
%     \cdbracket[Optionen per Key-Value Pair]{<Breite des Klammer>}{<Höhe des Klammer>}
%
% kann eine Klammer mit einer bestimmten Breite und Höhe gezeichnet werden.
%
% Dieses Makro besitzt 2 erforderliche Elemente, welche bei jedem Aufruf mit angegeben
% werden müssen.
%
% Die Argumente haben dabei folgende Bedeutung.
%
%   1. Argument: Breite der erzeugten Klammer
%   2. Argument: Höhe der erzeugten Klammer
%
%
% Die wichtigen Formatierungsoptionen werden alle innerhalb des optionalen Argumentes mittels
% Key-Value Pairs bestimmt.
%
% Dabei stehen folgende Optionen zur Verfügung:
%
%     thick          Hier wird die Dicke der Linien bestimmt
%                    Standardwert: \boxlinewidth
%
%     mode           Hier wird die Ausrichtung der Klammer angegeben
%                    Mögliche Werte
%                        left    linke Klammer
%                        top     obere Klammer
%                        right   rechte Klammer
%                        bottom  untere Klammer
%                    Standardwert: left
%
%     color          Hier wird die Farbe der Linie angegeben
%                    Es sollten nur die folgenden Farben benutzt werden:
%                        seeblau100
%                        seeblau65
%                        seeblau35
%                        seeblau20
%                        black
%                        schwarz60
%                        schwarz40
%                        schwarz20
%                        schwarz10
%                    Standardwert: seeblau100
%
%     arrowleft      Hier wird angegeben, ob die Klammer am linken (oberen) Ende mit einem Pfeil enden soll
%                    Mögliche Werte:
%                        true
%                        false
%                    Standardwert: false
%
%     arrowright     Hier wird angegeben, ob die Klammer am rechten (unteren) Ende mit einem Pfeil enden soll
%                    Mögliche Werte:
%                        true
%                        false
%                    Standardwert: false
%
%     arrowmiddle    Hier wird angegeben, ob die Klammer in der Mitte einen weiteren Pfeil besitzen soll der in die
%                    andere Richtung der Klammer zeigt (wie beim Makro \block mit der Optione arrow bei outerbottom / outertop)
%                    Mögliche Werte:
%                        true
%                        false
%                    Standardwert: false

\begin{frame}[t]
\subsectionpage%
Die \textbf{Klammer-Elemente} welche ebenfalls auf Plakaten verwendet werden, können mit dem Makro\\%
\hskip 1cm \texttt{\textbackslash cdbracket[<Optionen>]\{<Breite>\}\{<Höhe>\}}\\%
eingesetzt werden. Mit den Optionen, welche Sie in der Tex und Style Datei finden, können wieder Formatierungen vorgenommen werden.\\[\baselineskip]%
\centering\cdbracket[thick=8pt, mode=bottom, arrowleft=true, arrowright=true, arrowmiddle=true]{10cm}{2cm}\\[2cm]%
\end{frame}

\begin{frame}[t]
\frametitle{Weitere Klammern}
\vskip 18pt
\begin{minipage}{0.245\textwidth}
\cdbracket[mode=left]{2cm}{8cm}
\end{minipage}%
\begin{minipage}{0.5\textwidth}
\centering\cdbracket[thick=4pt, mode=top, arrowleft=true, arrowright=true]{8cm}{2cm}
\end{minipage}%
\begin{minipage}{0.245\textwidth}
\hfill\cdbracket[mode=right]{2cm}{8cm}
\end{minipage}%

\end{frame}



\section{Aufzählungen}
\subsection{Itemize}

\begin{frame}[t]
\sectionpage
\subsectionpage
	\begin{itemize}
		\item Ich bin das erste Item
		\item Ich bin das zweite Item
		\begin{itemize}
			\item Ich bin das erste Sub-Item des zweiten Items
			\item Ich bin das erste Sub-Item des zweiten Items
			\begin{itemize}
				\item Ich bin das erste Sub-Sub-Item des zweiten Sub-Items und bin auch gaaaaaaaanz laaaaaaaaang
				\item Ich bin das zweite Sub-Sub-Item des zweiten Sub-Items
			\end{itemize}
		\end{itemize}
	\item Ich bin das dritte Item
	\item u.s.w.
\end{itemize}
\end{frame}

\subsection{Enumerate}
\begin{frame}[t]
\subsectionpage
	\begin{enumerate}
		\item Ich bin das erste Item
		\item Ich bin das zweite Item
		\begin{enumerate}
			\item Ich bin das erste Sub-Item des zweiten Items
			\item Ich bin das erste Sub-Item des zweiten Items
			\begin{enumerate}
				\item Ich bin das erste Sub-Sub-Item des zweiten Sub-Items
				\item Ich bin das zweite Sub-Sub-Item des zweiten Sub-Items
			\end{enumerate}
		\end{enumerate}
		\item Ich bin das dritte Item
		\item u.s.w.
	\end{enumerate}
\end{frame}




\section{Weitere Hinweise}

\begin{frame}[t]
	\sectionpage
	\begin{itemize}
		\item Zum Übersetzen muss \texttt{XeLaTeX} verwendet werden, da somit die Systemschrift Arial eingebunden werden kann.
		\item Das Seitenverhältnis sollte nicht über \texttt{aspectratio}, sondern mittels \texttt{\textbackslash format} in der Präambel angepasst werden. Nur dadurch kann die richtige Auswahl der Schriftgrößen und die Platzierung der Elemente gewährleistet werden. Mehr dazu in den Dateien \texttt{benutzerhandbuch.tex} und \texttt{beamerthemeKonstanz.tex}. Dies wird jedoch in eine der nächsten Version an \texttt{aspectratio} von Beamer angepasst.	
	\end{itemize}
\end{frame}


\begin{frame}[t]
	\frametitle{Zusätzlich benötigte Pakete}
	\begin{itemize}
	    \item beamer
	    \item xcolor
		\item textpos
		\item soul
		\item ulem
		\item cmbright
		\item fontspec
		\item calc
		\item tikz
		\item keycommand
		\item ifthen
	\end{itemize}
\end{frame}

% Alle benötigten Pakete:
%  - beamer
%  - xcolor
%  - geometry
%  - babel
%  - fontenc
%  - textpos
%  - xunicode
%  - soul
%  - ulem
%  - ifthen
%  - keycommand
%  - tikz
%  - calc
%  - cmbright
%  - fontspec

\section{ToDo}

\begin{frame}[t]
	\sectionpage
	\begin{itemize}
		\item Zweitlogo bzw. Unterlogo
		\item Anpassung der \texttt{geometry} an \texttt{aspectratio}
		\item Weitere Beispiele (Tabellen, Biblografie, \ldots)
	\end{itemize}
\end{frame}

\end{document}