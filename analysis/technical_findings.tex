\chapter{Technical Findings}
At the beginning of this Chapter I want to clarify which quantities are measured. Then I will explain how the used measurement method works and how the signal processing is performed. After that I present what I did to optimize the signal. Later I will show what effect the ambient temperature has on the measurement. Further, I want to present the application of frequency domain reflectometry to find defects in the cabling. Finally, I will discuss the sample temperature dependence of the transmission.

Now we want to understand the measuring quantities of the vector network analyzer (VNA). It has two ports $i,j \in \{1,2\}$. Therefore, one can measure four so-called scattering parameters $S_{ij}$, which are connecting the incident\footnote{Commonly the outgoing and incident waves are defined in respect to the device under test. So $a$ is the output of the VNA, whereas $b$ is the measured wave amplitude at the VNA.} wave amplitudes $a$ with the outgoing wave amplitudes $b$, by
\begin{align}
    \left(\begin{array}{c}
         b_1\\
         b_2
    \end{array}\right)
    =
    \left(\begin{array}{cc}
         S_{11}&S_{12}\\
         S_{21}&S_{22}
    \end{array}\right)
    \cdot
    \left(\begin{array}{c}
         a_1\\
         a_2
    \end{array}\right)\,.
\end{align}

In general, we are interested in the transmission. Therefore, we will concentrate further in the forward transmission scattering parameter, defined by
\begin{align}
    S_{21}=\frac{b_2}{a_1}=|S_{21}| e^{\ima\phi_{21}}\,.
\end{align}
Since we compare two wave amplitudes we can calculate the respective maximum amplitude $|S_{21}|$ and the phase difference $\phi_{21}$ as well.

When we want to do further calculus we always have to keep in mind that $S_{21}$ is complex. So the absolute value and phase are given by
\begin{align}
    |S_{21}|&=\sqrt{\operatorname{Re}(S_{21})^2+\operatorname{Im}(S_{21})^2}\\
    \phi_{21}&=\operatorname{atan2}\left(\operatorname{Re}(S_{21}),\operatorname{Im}(S_{21})\right)\,.
\end{align}
The $\operatorname{atan2}$ operator is an extension of the inverse angle function $\operatorname{atan}$. If two Cartesian coordinates are passed as arguments to the $\operatorname{atan2}$ operator, the polar angle which is located in the correct quadrant is obtained. In consequence, the phase is given in a range of values from $0$ to $2\pi$.

When it comes to visualizing the scattering parameter, we will usually choose to plot the forward transmission $T_{21}$ in decibel (dB). It is calculated by
\begin{align}
    T_{21}=10\cdot \log_{10} \left( |S_{21}|^2 \right)
          =20\cdot \log_{10} \left( |S_{21}|\right)\,.
\end{align}

In the following, I am interested in the transmission depending on the angular frequency $\omega$ and magnetic field $H$. Since it is more convenient in technical applications to use the frequency $f$ with dimension Hertz (Hz), henceforth I will use $\omega/2\pi$, with dimension Hz\,\footnote{Since the typical frequencies are in the gigahertz regime, I will use GHz as unit.}. Likewise it is more convenient in technical applications, to express the magnetic flux density of a magnet in Tesla (T). Therefore, I will express the magnetic field $H$ in terms of $\mu_0H$, with unit T. \cite{ZNB40manual}

%%%%%%%%%%%%%%%%%%%%%%%%%%%%%%%%%%%%%%%%%%%%%%%%%%%%%%%%%%%%%%%%%%%%%%%%%%%%%%%%%%%%%%%%%%%%%%%
%%% Data processing
%%%%%%%%%%%%%%%%%%%%%%%%%%%%%%%%%%%%%%%%%%%%%%%%%%%%%%%%%%%%%%%%%%%%%%%%%%%%%%%%%%%%%%%%%%%%%%%
\section{Data Processing} \label{sec:data_processing}
%{\color{antiseeblau100}}
Now we want to understand how the measured scattering parameter is related to the susceptibility and how we can make the ferromagnetic resonance (FMR) observable.

First, we consider the raw measured scattering parameter $S_{21}$ as voltage ratio
\begin{align}
    S_{21}=\frac{V_\text{sys}+V_\text{ind}}{V_0}\,.
\end{align}
The voltage in the denominator $V_0$ is applied by the VNA, whereas the voltage in the numerator is a superposition of a systematic voltage $V_\text{sys}$ and of a CPW induced voltage $V_\text{ind}$.
The induced voltage is given by
\begin{align}
    V_\text{ind}=\ima\omega A\cdot V_0 \cdot \chi(\omega,H)\,,
\end{align}
where $A$ is a real-valued scaling parameter, that is dependent on the geometric dimensions of the co-planar waveguide (CPW) and the impedance of the entire system.

The systematic voltage is caused by the dielectric properties of the setup and depends only on the frequency. However, this voltage is much larger than the induced voltage. Since the FMR only affects the induced voltage, we have to consider it in the absence of the systematic voltage. That can be achieved by normalizing the measurement data by the normalization scattering parameter
\begin{align}
    S_{21}^0=\frac{V_\text{sys}}{V_0}\,.
\end{align}
With that in mind, we can write the normalized scattering parameter as
\begin{align}
    S_{21}^\text{n}=\frac{S_{21}}{S_{21}^0}=1+\frac{V_\text{ind}}{V_\text{sys}}\approx 1+\frac{V_\text{ind}}{V_0}\,.
\end{align}
Considering the induced voltage is way smaller than the voltage applied by the VNA $V_\text{ind}\ll V_0$ we can approximate $V_\text{ind}/V_\text{sys}\approx V_\text{ind}/V_0$.

Finally, our normalized scattering parameter is related to the susceptibility by
\begin{align}
    S_{21}^n=1+\ima\omega A\cdot \chi(\omega,H)\,.
\end{align}
Thus, the normalized scattering parameter no longer depends on the systematic voltage $V_\text{sys}$ or the voltage $V_0$ applied by the VNA. \cite{Louis2016}

Now that we understand how we relate the normalized scattering parameter to the susceptibility, we can address the signature of the FMR signal. From Chapter \ref{sec:theo_Kittel} we already know that the FMR frequency for an in-plane magnetized film is finite even in the absence of an external magnetic field. For applied fields the FMR frequency increases continuously. If negative fields are included, an axial symmetry to the zero-field line results. This leads to a typical curved \rotatebox[origin=c]{270}{$\pmb{\succ}$}-shape around the zero-field line, which becomes straighter for higher fields. The transmission decreases in the presence of the FMR.

The simplest method to determine the normalization scattering parameter is to find a frequency curve at a fixed magnetic field, which does not carry a FMR signal with certainty. Therefore, the maximum measured magnetic field $H_\text{n}$, where the FMR frequency is out of the measurement range is usually used for normalization. The normalization scattering parameter $S_{21}^0$ is then given by
\begin{align}
    S_{21}^0(\omega)=S_{21}(\omega, H_\text{n})\,.
\end{align}

The normalized scattering parameter $S_{21}^\text{n}$ and the normalized transmission $T_{21}^\text{n}$ are calculated by\footnote{Always keep in mind, that algebraic operations looks different for $T_{21}$ than for $S_{21}$, since they are logarithmic quantities.}
\begin{align}
    {S}_{21}^\text{n}&=\left.{S_{21}(\omega,H)}\,\middle/\, {S_{21}^0(\omega)}\right.\,,\label{eq:normS21}\\
    {T}_{21}^\text{n}&=T_{21}(\omega,H)-T_{21}^0(\omega)\,.
\end{align}

\begin{figure}
    \centering
    \import{analysis/really_small}{really_small.pgf}
    \caption[Raw and normalized transmission of sample CPW2]{Transmission $T_{21}$ of sample CPW2 depending on the frequency $\omega/2\pi$ and magnetic field $\mu_0H$. In \textbf{\color{antiseeblau100}a}, you can see the transmission $T_{21}^0$ for a fixed field at $H_\text{n}=1.1\,$T. In \textbf{\color{antiseeblau100}b} you can see a colorless grey scale map of the raw transmission $T_{21}$. In \textbf{\color{antiseeblau100}c} you can see a false color map of the normalized transmission $T_{21}^\text{n}$.}
    \label{fig:analysis_really_small}
\end{figure}
Figure \ref{fig:analysis_really_small} shows data obtained of sample CPW2 as an example for the normalization of the transmission. In panel a you can find the normalization transmission $T_{21}^0$, in panel b the raw transmission $T_{21}$ and in panel c the normalized transmission $T_{21}^\text{n}$. The \rotatebox[origin=c]{270}{$\pmb{\succ}$}-shaped FMR signal is clearly observable and not present at maximum field. At this point, I would like to emphasize the large amplitude ratio between raw and normalized transmission. The data range of the raw transmission covers approximately $60\,$dB, whereas the normalized transmission covers a data range of approximately $0.06\,$dB.

Now we want to turn to the quantitative evaluation of the FMR. In the following, I present an algorithm that provides reliable minima in the normalized transmission despite strong noise. The algorithm was applied to measurements of samples CPW2 and CPW3 and is visualized in Figure \ref{fig:analysis_kittel}.
\begin{figure}
     \centering
     \begin{subfigure}{\textwidth}
         \centering
         \import{analysis/kittel}{kittel_CPW2.pgf}\\
         \small\textbf{(1)} Sample: CPW2\\
         % [-1.48155226e+00  1.59066864e+04] [  3.82761252 426.54197394]
     \vspace{5mm}
     \end{subfigure}
     \begin{subfigure}{\textwidth}
         \centering
    \import{analysis/kittel}{kittel_CPW3.pgf}\\
         \small\textbf{(2)} Sample: CPW3\\
     \vspace{5mm}
         % [  32.43672464 9419.27273523] [ 12.12654013 586.13978913]
     \end{subfigure}
    \caption[Normalized transmission and minima of CPW2 and CPW3]{Normalized transmission and minima as a function of frequency $\omega/2\pi$ and magnetic field $\mu_0H$. The measurement of sample CPW2 is shown in \textbf{(1)} and of sample CPW3 is shown in \textbf{(2)}. Normalized transmission (false color) and minima found (\textbf{\color{antiseeblau100}magenta}) are shown in \textbf{\color{antiseeblau100}d}. Masked data areas are decolorized. In \textbf{\color{antiseeblau100}b}, the normalization transmission $T_{21}^0$ in \textbf{\color{seeblau100}blue} for $H_\text{n}=1\,$T can be seen. In \textbf{\color{antiseeblau100}a} and \textbf{\color{antiseeblau100}c}, the determined median curves $\langle\widetilde{T}_{21}^\text{n}(\omega)\rangle_H$ and $\langle\widetilde{T}_{21}^\text{n}(H)\rangle_\omega$ are plotted in \textbf{\color{seeblau65}light blue}.}
    \label{fig:analysis_kittel}
\end{figure}
First, I uniformly resized the datasets to a window from $0$ to $1\,$T and reduced the magnetic field resolution from $0.5\,$mT steps to $5\,$mT steps by averaging. Dataset CPW3 was cut off below $5\,$GHz, because there is no signal. Additionally, for this dataset, the frequency resolution was reduced from $20\,$MHz to $250\,$MHz by averaging. Now both datasets have the same resolution in frequency and magnetic field direction. Also the considered magnetic field section and the maximum frequency is the same. By averaging, some noise could already be reduced.

The normalized transmission $T_{21}^\text{n}$ is calculated as before using the normalization transmission $T_{21}^0$ at $H_\text{n}=1\,$T. Then I additionally normalize the transmission line-by-line to a range from $0$ to $1$ in arbitrary units\footnote{Since I am not working with quantitatively comparable transmissions, I use again $\widetilde{T}_{21}^\text{n}$ as notation}. Further, I form the median row-by-row $\langle\widetilde{T}_{21}^\text{n}(\omega)\rangle_H$ and subtract it from each row. These additional normalizations allow an effective reduction of frequency dependent noise. In magnetic field direction already subtracting the column-wise formed median $\langle\widetilde{T}_{21}^\text{n}(H)\rangle_\omega$ achieves a sufficiently good reduction of magnetic field noise. Then I mask frequency bands that are still highly noisy. Additionally, I mask data that are far away from the visible FMR signal.

Now minima are determined row-by-row and column-by-column. Its advantage is that many minima are obtained. The disadvantage is that some minima are lined up vertically or horizontally. Nevertheless, to prevent excessive weighting for horizontal or vertical features, I join the minima row-by-row and column-by-column. This means, e.g. that several minima obtained row-by-row at the same magnetic field are reduced to one minimum by averaging. The magnetization properties, which can be determined from the obtained minima, are discussed in Chapter \ref{chapter:physical_findings}.

%%%%%%%%%%%%%%%%%%%%%%%%%%%%%%%%%%%%%%%%%%%%%%%%%%%%%%%%%%%%%%%%%%%%%%%%%%%%%%%%%%%%%%%%%%%%%%%
%%% Modelling
%%%%%%%%%%%%%%%%%%%%%%%%%%%%%%%%%%%%%%%%%%%%%%%%%%%%%%%%%%%%%%%%%%%%%%%%%%%%%%%%%%%%%%%%%%%%%%%
\section{Modeling}
In this Section I will take the opportunity to present several other background treating methods, even though they did not yield better results than the already described method.

In another approach, instead of a fixed-field reference, as used in Section \ref{sec:data_processing}, a {moving-field reference} is used. We consider a small frame around the theoretically predicted resonance frequency $\omega_\text{res}^\text{theo}(H)$, with width $\Delta\omega_\text{f}$. So, we evaluate the scattering parameter solely in this field, given by
\begin{align}
    S_{21}^\text{f}=S_{21}(\omega_\text{res}^\text{theo}(H)\pm\Delta\omega_\text{f},H)\,.
\end{align}

As in Section \ref{sec:data_processing} discussed, I took the highest field within the given frame to obtain a reference curve. The normalization scattering parameter is given by
\begin{align}
    S_{21}^{\text{f},0}=S_{21}(\omega,H_\text{n})\,.
\end{align}

The normalized scattering parameter is calculated as before by
\begin{align}
    S_{21}^{\text{f},\text{n}}=S_{21}^\text{f}/S_{21}^{\text{f},0}\,.
\end{align}
You can see an example for the moving-field reference in Figure \ref{fig:analysis_mod}.3, right next to a fixed-field reference background treatment in Figure \ref{fig:analysis_mod}.1. \cite{maierflaig2017,siegl2020}
\begin{figure}
     \centering
     \begin{turn}{90}
     \begin{subfigure}[b]{3.65in}
         \hspace{.2in}
         \import{analysis/modelling}{dd.pgf}\\
         \centering\small\textbf{(2)} Derivative divide $\operatorname{d}_\text{D}S_{21}$\\
         %\caption{Derivative divide $\operatorname{d}_\text{D}S_{21}$}
         %\label{fig:analysis_mod_dd}
     \end{subfigure}
     \end{turn}
     \hspace{.2in}
     \begin{turn}{90}
     \begin{subfigure}[b]{3.65in}
         \hspace{.2in}
         \import{analysis/modelling}{U.pgf}\\
         \centering\small\textbf{(4)} Microwave permeability parameter $U(\omega)$\\
         %\caption{Microwave permeability parameter $U(\omega)$}
         %\label{fig:analysis_mod_U}
     \end{subfigure}
     \end{turn}
     \begin{turn}{90}
     \begin{subfigure}[b]{4in}
         \centering
         \import{analysis/modelling}{raw.pgf}\\
         \centering\small\textbf{(1)} Fixed-field reference method\\
         %\caption{Fixed-field reference method}
         %\label{fig:analysis_mod_raw}
     \end{subfigure}
     \end{turn}
     \hspace{.2in}
     \begin{turn}{90}
     \begin{subfigure}[b]{4in}
         \centering
         \import{analysis/modelling}{mrfm.pgf}\\
         \centering\small\textbf{(3)} Moving-field reference method\\
         %\caption{Moving-field reference method}
         %\label{fig:analysis_mod_mrfm}
     \end{subfigure}
     \end{turn}
    % \rotatebox[origin=c]{180}{$\omega/2\pi$ [GHz]}
    % $\mu_0H$ [T]
    % \rotatebox[origin=c]{180}{$\operatorname{Re}\left(\operatorname{d}_\text{D}S_{21}\right)\cdot 20$}
    % \rotatebox[origin=c]{180}{$\operatorname{Im}\left(U(\omega)\right)\cdot 10^4$}
    % \rotatebox[origin=c]{180}{$T_{21}^\text{n}$ [$0.01\,$dB]}
    % \rotatebox[origin=c]{180}{$T_{21}^\text{f,n}$ [$0.01\,$dB]}
    % \rotatebox[origin=c]{180}{$T_{21}^\text{f,0}$ [dB]}
    % $H_\text{n}=1.00\,$T
    % $H_\text{n}(\omega_\text{res}^\text{theo})$
    \caption[Comparison between different noise reduction methods for transmission data of sample CPW2]{Comparison between different signal enhancement methods for transmission data of sample CPW2, depending on the frequency $\omega/2\pi$ and the magnetic field $\mu_0H$. In \textbf{(1)} and \textbf{(3)} the two normalization methods fixed-field reference and moving-field reference are shown. In \textbf{\color{antiseeblau100}a} the normalization transmission is shown and in \textbf{\color{antiseeblau100}b} the normalized transmission. Additionally, the course of the magnetic field ${H}_\text{n}$ selected for normalization is shown in \textbf{\color{antiseeblau100}magenta}. \textbf{(2)} shows the real part of the derived scattering parameter $\operatorname{Re}\left(\operatorname{d}_\text{D}S_{21}\right)$. \textbf{(4)} shows the imaginary part of the microwave permeability parameter $\operatorname{Im}\left(U(\omega)\right)$.}
    \label{fig:analysis_mod}
\end{figure}

In comparison to the fixed-field method, the moving-field reference method has a slight advantage when it comes to line-by-line averaging. The drifts within one line are not that pronounced in comparison to the fixed-field normalization. You can see an example for line-by-line averaging in Figure \ref{fig:analysis_mod_avg}, in the Appendix. Due to the complexity of calculating the moving-field this method is not used further.

Next, I want to present the so-called 'derivative divide' method by Hannes Maier-Flaig \cite{maierflaig2018, maierflaig2017}. This method is particularly suitable for out-of-plane measurements, because then the FMR signal is strictly linear as function of the magnetic field. Even if this is true for high magnetic fields and in the in-plane case, this is not the case for smaller magnetic fields. Therefore, the application is not quite suitable, even though the method removes interfering temperature drifts and frequency dependencies without prior system calibration. The basic idea is to calculate the numerical deviation of the scattering parameter by
\begin{align}
    \operatorname{d}_\text{D}S_{21}=\frac{S_{21}(\omega,H+\Delta H_\pm)-S_{21}(\omega,H-\Delta H_\pm)}{S_{21}(\omega,H)\cdot\Delta H_\pm}\,,
\end{align}
where $\Delta H_\pm$ is the step width. We can easily show that $\operatorname{d}_\text{D}S_{21}=\operatorname{d}_\text{D}S_{21}^\text{n}$ holds true. Therefore, it makes no difference whether the raw or normalized scattering parameter is used to calculate the numerical derivative.

However, the deviated scattering parameter is related to the susceptibility by 
\begin{align}
    \operatorname{d}_\text{D}S_{21}&=\ima\omega A'\  \frac{\chi(\omega+\Delta\omega_\pm)-\chi(\omega-\Delta\omega_\pm)}{2\Delta\omega_\pm}\,,
\end{align}
where $\Delta\omega_\pm$ is the frequency step, given by approximately $\Delta\omega_\pm\approx\gamma\mu_0\Delta H_\pm$. You can find an application example in Figure \ref{fig:analysis_mod}.2. In advance the resolution of the data was strongly reduced by averaging in order to achieve the best visible results.

Finally, there is also the uncalibrated effective {microwave permeability} parameter
\begin{align}
    U(\omega)=\pm \ima \, \frac{\operatorname{ln}\left( S_{21}^n(\omega,H)\right)}{\operatorname{ln}\left( S_{21}^0(\omega,H)\right)}\,.
\end{align}
The sign is chosen in the way, that $\operatorname{Im}(U(\omega))$ is negative in the vicinity of the FMR peak. \cite{Kalarickal2006, Barry1986}

The susceptibility can be fitted by
\begin{align}
    U_\text{fit}(\omega)=C\left(1+\chi_0+\chi(\omega) e^{\ima\phi} \right)\,,
\end{align}
where $C$ is a real-valued scaling parameter, $\chi_0$ is a complex offset and $\phi$ is a phase shift. You can see an application example in Figure \ref{fig:analysis_mod}.4. \cite{maierflaig2018, maierflaig2017}

All these modeling approaches try to reduce background noise, depending on the frequency, but also apparently on the field. It is quite obvious that there is still significant noise left in the data. Therefore, I will discuss further signal optimization in the next Section and the obtained magnetization properties in Chapter \ref{chapter:physical_findings}. 

%%%%%%%%%%%%%%%%%%%%%%%%%%%%%%%%%%%%%%%%%%%%%%%%%%%%%%%%%%%%%%%%%%%%%%%%%%%%%%%%%%%%%%%%%%%%%%%
%%% Signal Optimization
%%%%%%%%%%%%%%%%%%%%%%%%%%%%%%%%%%%%%%%%%%%%%%%%%%%%%%%%%%%%%%%%%%%%%%%%%%%%%%%%%%%%%%%%%%%%%%%
\section{Signal Optimization}
In this Section I would like to present chronologically various results of my work that have a strong technical aspect and deal with signal optimization. First, I will start with the HelioxVL setup and present my findings on the hysteresis of the setup. Then the results obtained by this setup and the BlueFors setup will be compared. Afterwards, datasets that were recorded either in the frequency direction or in the magnetic field direction will be compared. Finally, the parameters I used will be discussed and recommendations for future measurements will be given.

%%%%%%%%%%%%%%%%%%%%%%%%%%%%%%%%%%%%%%%%%%%%%%%%%%%%%%%%%%%%%%%%%%%%%%%%%%%%%%%%%%%%%%%%%%%%%%%
%%% HelioxVL Hysteresis
\subsection{HelioxVL Setup Magnetization} \label{sec:analysis_hysteresis}
To gain a better understanding of the signals obtained in the HelioxVL setup, I performed several initial measurements. To this end, I performed fast, low-resolution measurements on sample CPW1.
Another aspect of these studies addresses the role of the field sweep directions and their possible advantages.

The results with the FMR typical \rotatebox[origin=c]{270}{$\pmb{\succ}$}-shape can be examined in Figure \ref{fig:analysis_hysteresis}. 
\begin{figure}
    \centering
    \import{analysis/hysteresis}{hysteresis.pgf}
    % {\color{seeblau100}$\pmb{\leftarrow}$} $\mu_0H$ [T]
    \caption[Normalized transmission data in down and up sweep direction]{Normalized transmission $T_{21}^\text{n}$ in down ({\color{seeblau100}$\pmb{\leftarrow}$}, \textbf{\color{antiseeblau100}a}) and up ({\color{seeblau100}$\pmb{\rightarrow}$}, \textbf{\color{antiseeblau100}b}) sweep direction in dependence of frequency $\omega/2\pi$ and magnetic field $\mu_0H$ of CPW1 measured at the HelioxVL setup, initially magnetized at $3\,$T. In addition, dashed lines in \textbf{\color{antiseeblau100}magenta} have been added for visual emphasis.}
    \label{fig:analysis_hysteresis}
\end{figure}
After a $+3\,$T initialization I started the measurement first at $+320\,$mT (down sweep, {\color{seeblau100}$\pmb{\leftarrow}$}, panel a) and then at $-320\,$mT (up sweep, {\color{seeblau100}$\pmb{\rightarrow}$}, panel b).

It is easy to see that the center of the \rotatebox[origin=c]{270}{$\pmb{\succ}$}-shaped FMR signal is shifted by $+4\,$mT towards zero in the down sweep, whereas in the up sweep the \rotatebox[origin=c]{270}{$\pmb{\succ}$}-shape is exactly occurring at $0\,$mT. It is conceivable that the magnet has a small offset, but as well that components of the cryostat can be magnetized. A frequency independent feature can be observed in both measurements. In the case of the up sweep, it is at $+8\,$mT and in the the down sweep at $-1\,$mT. What is the cause of this feature has not been investigated further. However, this also speaks for a hysteretic magnetization effect. We were able to determine that the SK connector of the measurement line are ferromagnetic and are very likely responsible for this hysteric magnetization.
%Since the volume of the examined Co is simply too small to induce such magnetization effects, this effect must come from the setup itself. It is known that the magnets used can also have a small offset. However, we were able to determine that the SK connector of the measurement line are ferromagnetic and are very likely responsible for this hysteric magnetization.

For future measurements initialization has been omitted, as I have also measured with way stronger magnetic fields. I assume that magnetic fields around $1\,$T are sufficient to neglect any magnet training effects. So, inherent down sweeps from the initially applied magnetic field are measured. Of course this can still be varied in sign.

%%%%%%%%%%%%%%%%%%%%%%%%%%%%%%%%%%%%%%%%%%%%%%%%%%%%%%%%%%%%%%%%%%%%%%%%%%%%%%%%%%%%%%%%%%%%%%%
%%% Setup comparison
\subsection{Comparison of the Setups Used}
Finally, one focus of my work was to reproduce the FMR measurements in the HelioxVL setup. This measurements were already done previously in the BlueFors setup by Sergej Andreev. To this end, I wanted to use a sample with the strongest potential signal. Since this depends on the volume, I used CPW1 already described in Section \ref{sec:sample_design}.

Even though both measurements use very different measurement parameters and different sweep types, they can be compared qualitatively. The measurement at the BlueFors was done with Sergej Andreev's measurement program. This measurement method is explained in detail in Section \ref{sec:hf_sweep}. 

Panel a of Figure \ref{fig:analysis_setup_comp} shows the normalization transmissions, obtained at a magnetic field $H_\text{n}=315\,$mT.
\begin{figure}
    \centering
    \import{analysis/setup_comp}{setup_comp.pgf}
    \caption[Normalized transmission of sample CPW1 at the setups HelioxVL and BlueFors]{Normalized transmission $\widetilde{T}_{21}^\text{n}$ of the same sample CPW1 in the setups HelioxVL (\textbf{\color{antiseeblau100}b}) and BlueFors (\textbf{\color{antiseeblau100}c}) in dependence of frequency $\omega/2\pi$ and magnetic field $\mu_0H$. In panel \textbf{\color{antiseeblau100}a} the normalization transmissions $\widetilde{T}_{21}^0$ in \textbf{\color{antiseeblau100}magenta} and \textbf{\color[rgb]{0.878733401483036, 0.49928686449731624, 0.7675654756534815}light magenta} at $H_\text{n}=315\,$mT are shown.}
    % H=.315
    \label{fig:analysis_setup_comp}
\end{figure}
In magenta, the raw transmission measured in the HelioxVL setup and in light magenta, the already zero-field calibrated transmission measured in the BlueFors setup is shown. The normalized transmission measured in the HelioxVL and BlueFors setup is shown in panel b and c. 

Even though both measurements are different in noise shape and noise intensity due to the different parameter and sweep types, the \rotatebox[origin=c]{270}{$\pmb{\succ}$}-shape typical for FMR is recognizable. Despite all this, it can be said with certainty that FMR measurements can be made on both setups.

%%%%%%%%%%%%%%%%%%%%%%%%%%%%%%%%%%%%%%%%%%%%%%%%%%%%%%%%%%%%%%%%%%%%%%%%%%%%%%%%%%%%%%%%%%%%%%%
%%% hf sweep
\subsection{Magnetic Field and Frequency Sweep method} \label{sec:hf_sweep}
Usually, the VNA measures the scattering parameters for all set frequencies in a single sweep\footnote{Even though this is called a sweep, the VNA measures per default in a so called stepped method. Here each frequency is set before measurement and is changed just after the measurement. \cite{ZNB40manual}}. Therefore, it seems only natural to set a fixed magnetic field and then measure a frequency sweep. In the following, I will refer to this procedure as frequency sweep method.

However, also with a fixed frequency, the magnetic field can be swept. Henceforth, I will call this method the magnetic field sweep method. This method was previously implemented by Sergej Andreev and measured with his measuring program.

The measurement procedure of the magnetic field sweep method differs from the frequency sweep method in that a zero-field calibration is carried out first. At this point, the magnet is given a sweep rate and target. As the magnetic field changes continuously, the VNA measures miniate frequency sweeps, with a small span around a center frequency.  The collected data, magnetic field, frequency, transmission and phase are then averaged to obtain a data point. For visualization, a binning algorithm is also used later, in which transmission and phase are averaged again.

A problem that should not be underestimated is the averaging. As I mentioned in the beginning, the transmission is a logarithmic quantity. Therefore the scattering parameter needs to be used to calculate a mean value instead and calculate the transmission from that, as
\begin{align}
    T_{21}\left(\overline{S_{21}}\right) &= 20\cdot \log_{10} \left( |\overline{S_{21}}|\right) \\
    &= 20\cdot \log_{10} \left( \left|\frac{1}{N}\sum_{i=0}^N{S_{21,i}}\right|\right)\,.
\end{align}

However, calculating the mean of the transmission directly, can be traced back to a geometric mean of the magnitude of the scattering parameter, like
\begin{align}
    \overline{T_{21}}&= \overline{20\cdot \log_{10} \left( \left|{S_{21}}\right|\right)}\\
    &= \frac{20}{N}\cdot \sum_{i=0}^N\log_{10} \left( \left|{S_{21}}\right|\right)_i\\
    &= 20 \cdot \log_{10} \left( \prod
    _{i=0}^N\sqrt[N]{\left|{S_{21}}\right|_i}\right)\,.
\end{align}

Fortunately, the transmission values within a miniate sweep and between neighbouring data points do not differ much. Considering this, the geometric mean and arithmetic mean are quite close to each other. Therefore, it is still possible to discuss semi-quantitatively. Larger differences can only be found, if there are strong outliers among the values, which is why the geometric mean is often used as a noise filter. I will use $\widetilde{T}_{21}$ when the geometrically and arithmetically averaged transmission values are compared. \cite{gonzalez2008}

Superconducting magnets, as we use them, can only be swept very slowly. This results in a disadvantage of the magnetic field sweep method with respect to the measurement speed. Thus, at a constant frequency the field is swept up at maximum rate and then swept down again. During this process it is possible to measure well. However, there is no great information gain to measure up and down sweep each time.

In Figure \ref{fig:analysis_hf_sweep} a comparison of two measured down sweeps is made between frequency sweep method (panel b) and with magnetic field sweep method (panel c). 
\begin{figure}
    \centering
    \import{analysis/hf_sweep}{hf_sweep.pgf}
    \caption[Normalized transmission of sample CPW2 in frequency and magnetic field sweep method]{Normalized transmission $\widetilde{T}_{21}^\text{n}$ of the same sample CPW2 at the BlueFors setup in frequency sweep method (\textbf{\color{antiseeblau100}a}) and magnetic field sweep method (\textbf{\color{antiseeblau100}b}) in dependence of frequency $\omega/2\pi$ and magnetic field $\mu_0H$. Panel \textbf{\color{antiseeblau100}a} shows the normalization transmissions $\widetilde{T}_{21}^0$ in \textbf{\color{antiseeblau100}magenta} and \textbf{\color[rgb]{0.878733401483036, 0.49928686449731624, 0.7675654756534815}light magenta} at $H_\text{n}=1.05\,$T.}
    % H=1.05
    \label{fig:analysis_hf_sweep}
\end{figure}
The up sweep dataset of the magnetic field sweep method is available, but does not provide any additional relevant information and is therefore not shown. 

So, for each frequency a whole up and down sweep of the magnetic field has to be repeated. With limited measurement time the frequency resolution is therefore severely limited. Thus, in Figure \ref{fig:analysis_hf_sweep}c a frequency step size of $250\,$MHz can be seen, with a total measurement time of about $200\,$h.

With the frequency sweep method, on the other hand, there are more options regarding the parameters. A higher frequency resolution of $20\,$MHz is usually used. However, the two measurements seen in Figure \ref{fig:analysis_hf_sweep} were made with the same parameters such as resolution and range. By repeatedly averaging the same frequency sweep at the same field, the frequency sweep method could be slowed down to a total measurement time of about $75\,$h for a down sweep dataset. With well comparable FMR signals between magnetic field sweep method and frequency sweep method the frequency sweep method is clearly faster.

Finally, I want to talk about another not that obvious advantage of the frequency sweep method. When it comes to noise apparently the frequency sweep measurement in Figure \ref{fig:analysis_hf_sweep}b has additional noise. In both measurements horizontal lines can be seen, most likely due to frequency depending effects. However, in the frequency sweep method additionally vertical lines appear. Since we can exclude magnetic dependencies by the magnetic field sweep measurement, we have to assume, that some time dependent noise is disturbing the measurement. This time-dependent noise is included in both measurements but can only be resolved separately from frequency-dependent noise in the frequency sweep method. In the next Section, this time-dependent noise will be explained in more detail.

%%%%%%%%%%%%%%%%%%%%%%%%%%%%%%%%%%%%%%%%%%%%%%%%%%%%%%%%%%%%%%%%%%%%%%%%%%%%%%%%%%%%%%%%%%%%%%%
%%% Temperature induced noise
\subsection{Noise Induced by the Laboratory Temperature} \label{sec:analysis_temp}
In this Section I would like to talk about temperature induced noise. I could show that the room temperature in laboratory P10 varies up to $2\,^\circ$C. This significantly disturbs the BlueFors measurement electronics, consisting of VNA and pre-amplifier.

Essentially, I made three measurements to show the laboratory temperature induced noise. The results of these measurements are shown in Figure \ref{fig:analysis_temp}. 
\begin{figure}
    \centering
    \import{analysis/temp}{temp.pgf}
    % $T_\text{RT}$ [$^\circ$C]
    % $\widetilde{T}_{21}^\text{n}$ [$0.01\,$dB]
    \caption[Normalized transmission over frequency and as well as temperature over time]{Normalized transmission $T_{21}^\text{n}$ over frequency $\omega/2\pi$ and as well as temperature over time $t$. The temperature is either measured in the laboratory $T_\text{RT}$ or the box $T_\text{box}$, used for isolation. For normalization the last transmission $T_{21}^0(t_\text{max})$ was used. \textbf{\color{antiseeblau100}a}, \textbf{\color{antiseeblau100}b} \& \textbf{\color{antiseeblau100}c} are showing the measurements without, with and with tempered pre-amplifier.}
    \label{fig:analysis_temp}
\end{figure}

For measurement one I connected the output and input of the VNA and measured the transmission for all set frequencies, shown in Figure \ref{fig:analysis_temp}a. For measurement two I connected the pre-amplifier in between, shown in Figure \ref{fig:analysis_temp}b. Measurement three is shown in Figure \ref{fig:analysis_temp}c and is made with a temperature stabilization\footnote{The stabilization consists out of a temperature controller and a styrofoam box for isolation. The temperature controller was built by Sergej Andreev and consists out of a Peltier element and a water cooling system. This setup can control the pre-amplifier to stable $25\,^\circ$C.}, to control the temperature of the pre-amplifier. Simultaneously, I logged either the room temperature $T_\text{RT}$, or the temperature within the styrofoam box $T_\text{box}$, used for isolation. 

Besides a regular fluctuation of the room temperature in measurements one and two in a $0.5$ to $1\,$h cycle, a particularly strong fluctuation occurs every $10$ to $12\,$h. I guess this has to do with the air conditioning of the laboratory. Furthermore, a low frequent fluctuation with a period of $24\,$h can be observed. This means that the laboratory temperature depends slightly on the time of day. The high frequency fluctuations in measurement two are way more irregular in amplitude. This can be explained by the fact that the temperature is more stable at weekends (measurement one) compared to during the week (measurement two) with higher numbers of people in the laboratory.

Measurement one without a pre-amplifier shows that the transmission varies together with the temperature. To compare the fluctuation quantitatively I calculated the average value column-by-column and then calculated the standard deviation $\sigma$ (STD) of both the transmission and the temperature. The room temperature has a STD of $\sigma(T)=0.53\,^\circ$C. The transmission has a STD of $\sigma(\langle T_{21}\rangle_\omega)=3.9\cdot10^{-3}\,$dB. Calculating the ratio of these we get $\sigma(\langle T_{21}\rangle_\omega)/\sigma(T)=7.3\cdot10^{-3}\,$dB/$^\circ$C, which is even slightly better than the specified drift of $0.01\,$dB/$^\circ$C for the VNA. \cite{ZNB40brochure}

In measurements two and three, frequency bands with different temperature-dependent transmission characteristics appear. Especially in the higher-frequency third, temperature-dependent transmission fluctuations are clearly visible. Especially the day-dependent temperature fluctuations seem to have a strong effect on the transmission. The transmission in measurement two is significantly higher in the first $\approx15\,$h. This can be explained by a self-induced heating-up of the pre-amplifier to a stable operating temperature.

Quantitatively, the STD of transmission in the whole measurement two is $\sigma(\langle T_{21}\rangle_\omega)=4.7\cdot10^{-2}\,$dB, in the second half of measurement two the STD is $\sigma(\langle T_{21}\rangle_\omega)=1.3\cdot10^{-2}\,$dB and in measurement three the STD is $\sigma(\langle T_{21}\rangle_\omega)=1.6\cdot10^{-2}\,$dB. It would be interesting to know how much the laboratory temperature varied during measurements three, since the transmission varies more than in the second half of measurement two. However, it can be deduced from the box temperature $T_\text{box}$ that the cooling of the pre-amplifier works reliable. The temperature fluctuations are characterized by the course of the day, which suggests a cooling water temperature fluctuation.

Only from the transmission noise perspective the further use of the pre-amplifier is not recommended. Nevertheless, its usage is recommended especially at higher frequencies, since the signal gain of the pre-amplifier outweighs the higher noise. For comparison two measurements with and without tempered pre-amplifier are shown in Figure \ref{fig:analysis_temp_lowbw}, in the Appendix.

%%%%%%%%%%%%%%%%%%%%%%%%%%%%%%%%%%%%%%%%%%%%%%%%%%%%%%%%%%%%%%%%%%%%%%%%%%%%%%%%%%%%%%%%%%%%%%%
%%% How to parameter
\newpage
\subsection{Parameter Study} \label{sec:parameter_study}
In this Section I would like to make recommendations for the parameter space to be chosen. Often enough, the measurement time is limited and cannot be increased arbitrarily. Here I mainly refer to the magnets, the VNA and the pre-amplifier used.

Magnetic fields can be adjusted in steps of down to $0.5\,$mT and I would also recommend working with that stepsize. For fast measurements you can also increase to $1\,$mT steps. From Section \ref{sec:analysis_hysteresis} we know that asymmetries can arise around $0\,$T, so  measurements should also be made generously around $0\,$T. The maximum frequency used of the VNA is $40\,$GHz, with the gyromagnetic ratio of about $28\,$GHz/T, the upper FMR limit for nearly spherical geometries is about $1.4\,$T. In practice, it has turned out that only fields up to $1.1\,$T are interesting, since in-plane measurements are conducted. Summarizing, I would recommend a measurement between -$50\,$mT and $1.2\,$T in $0.5\,$mT steps, which corresponds to $2501$ measuring points in magnetic field direction. \cite{AM430manual,IPS12010manual}

The VNA provides a frequency range from $100\,$kHz to $40\,$GHz, which in general I would always use. For faster measurements the frequency can be cut off below the upper limit. Thus, the signal strength for high frequencies also decreases strongly, e.g. in Section \ref{sec:analysis_temp}, I only measured up to $26\,$GHz. In combination with the pre-amplifier measurements should only be taken from $2\,$GHz, as this only supports frequencies between $2$ to $50\,$GHz. The number of measurement points is almost arbitrary\footnote{Technically it is limited to $100001$ points by the VNA. However, you can split your sweep into several sub-sweeps. Doing so the number of points can increase significantly.}. Therefore, the increase in total measurement time has to be considered. In practice around $2000$ points have proven to be useful.
I would suggest measurements between $2\,$GHz and $40\,$GHz, in $20\,$MHz steps. That corresponds to $1901$ measuring points in frequency direction. \cite{ZNB40manual,U7227manual}

With an image resolution of $2501\times1901$ points, it is easy to zoom into features afterwards. Also, a smoother image can be achieved by averaging afterwards.

Unfortunately, no reliable numbers on the relationship between bandwidth and dynamic range can be found in the VNA specifications. However, it is certain that the smaller the bandwidth, the better the resolution for small signal strengths. However, with a smaller bandwidth the measurement time increases. Averaging over several frequency sweeps helps as well against noise, which also increases the measurement time. While in the end it is always a question of weighing measuring time against signal strength, in practice, it has been shown that a bandwidth of $1\,$kHz and $10$ averages yields satisfying results, when the pre-amplifier is used. If you want to measure without a pre-amplifier, a bandwidth of about $30\,$Hz and $3$ averages is recommended. In general, the use of the pre-amplifier is recommended, especially for frequencies above $30\,$GHz. \cite{ZNB40manual,ZNB40brochure,ZNB40spec}

%%%%%%%%%%%%%%%%%%%%%%%%%%%%%%%%%%%%%%%%%%%%%%%%%%%%%%%%%%%%%%%%%%%%%%%%%%%%%%%%%%%%%%%%%%%%%%%
%%% FDR
\begin{figure}[t]
    \centering
    \import{analysis/FDR}{fdr2.pgf}
    \caption[Reflection parameters as a function of resonator length]{Reflection parameters $\hat{R}_{11}$ \& $\hat{R}_{22}$ in panel \textbf{\color{antiseeblau100}a} \& \textbf{\color{antiseeblau100}b} as a function of resonator length $d$. Three measurements each at \textbf{\color[rgb]{0.000000,0.404312,0.680182}$\mathbf{4}$\,K}, \textbf{\color[rgb]{0.085109,0.628859,0.828930}$\mathbf{161}$\,K} and \textbf{\color[rgb]{0.499287,0.767565,0.878733}$\mathbf{285}$\,K} are shown. The maximum of the dominant wide peaks are marked in \textbf{\color{antiseeblau100}magenta} at $57.4\,$cm and $54.9\,$cm.}
    \label{fig:analysis_tdr}
    % $T_\text{FMR}=4\,$K
    % \,$54.9\,$cm
\end{figure}

\section{Frequency Domain Reflectometry}

Frequency domain reflectometry (FDR) is a commercially available, powerful software for analyzing transmission lines with a VNA. I utilized the measured reflection scattering parameters to find wiggle contacts by applying a fast Fourier transform myself. Since this technique was implemented by Smith-Rose in the 1930s, more sophisticated reconstruction algorithms have been put forward. Because a limited understanding of this technique will suffice in the following, I will not go into detail. More information can be found following sources, \cite{smithrose1933, vanhamme1990, shin2005, keysightFDR}.

The reflection scattering parameter $S_{ii}, i\in\{1,2\}$ is defined by
\begin{align}
    S_{ii}=\frac{b_i}{a_i}\,.
\end{align}

The VNA measures the scattering parameter as a function of frequency. From frequency domain to the time domain can be switched by applying the symmetrized ortho-normalized fast Fourier transformation
\begin{align}
    \hat{S}_{ii}(t)=\operatorname{FFT}\left( S_{ii}(f)\right)\,.
\end{align}

Now I rescale the time-axis to get the typical resonator length $d$, by
\begin{align}
    \Delta d = \frac{c_0/\sqrt{\varepsilon_\text{r}}}{2\cdot 2\pi\Delta f}\,.
\end{align}
Here is $c_0$ the speed of light and the factor of propagation velocity can be approximated as $1/\sqrt{\varepsilon_\text{r}}\approx 0.70-0.77$ for common microwave cables, where $\varepsilon_\text{r}$ is the relative permittivity. A factor of 2 is inserted, on account of the typical resonator length and not the path length is to be considered. \cite{HS2020, elspec}

Finally, the reflection parameter in decibels is calculated by
\begin{align}
    \hat{R}_{ii}(d)=20\cdot \log_{10}\left(|\hat{S}_{ii}(d)|\right)\,.
\end{align}

In the following, I will use an example to explain the possibility of finding a loose contact in the transmission line. The transmission has drastically decreased in the course of a cooling process. I have carried out three reflection measurements during the warm-up process. 

The first measurement was made at $4\,$K, the second at $161\,$K. The high temperature superconducting cables used in some sections are superconducting only up to about $75\,$K, see next Section \ref{sec:analysis_fT_map}. The transmission during the first two measurements were significantly worse than I would usually expect and leads me to the conclusion that the transmission line is interrupted. The last measurement was taken near room temperature at $285\,$K. At this temperature I was able to measure the expected transmission and I can assume that the transmission line is intact again.

The respective reflection parameters $\hat{R}_{11}$\,\&\,$\hat{R}_{22}$ as a function of resonator length $d$ can be seen in Figure \ref{fig:analysis_tdr}. The reflection parameters at $4\,$K and $161\,$K did not differ significantly from each other in their magnitudes and are referred as interrupted state in the following. The reflection parameters at $285\,$K in intact state differs significantly from the other reflection parameters measured.

The first two strong peaks are formed at resonator lengths corresponding to the distance between the $-10\,$dB dampers of the setup and the adjacent connectors. The second peak, like most of the following peaks, looks like double peaks. This can be explained by the placeholders that are built in instead of the original dampers. The placeholders and the next cable termination form two resonators of only slightly different length. The following peaks corresponding to resonator length up to $50\,$cm can all be explained by resonances in the different cable parts.

Around a resonator length of $57.4\,$cm, respectively $54.9\,$cm a broad peak builds up in the interrupted state measurements. This resonator length corresponds to the cable Sections between the sample holder and the mixing chamber (MXC). There is a significant difference in magnitude between the interrupted and intact state of the cables. From this difference I conclude that the interruption of the transmission line must be at the cable connection to the PCB\footnote{The inner conductor pressed onto the PCB board was stripped too tightly and pulled off the PCB board when it cooled down. I was able to fix this by stripping the inner conductor again generously.}. However, even in the intact state, the peaks corresponding to the same resonator length, are still higher than most of the peaks corresponding to smaller lengths. This is again to be expected since the inner conductors of the cable are simply pressed onto a PCB board in the sample holder, thus a higher reflection is assumed than within a commercial connector.
I attribute the peaks at higher resonator lengths, to higher resonance orders. 

In summary, I could show that with the measurement of both reflection scattering parameters in interrupted and intact state, a localization of a interruption in the transmission line is possible just by using the VNA.

%%%%%%%%%%%%%%%%%%%%%%%%%%%%%%%%%%%%%%%%%%%%%%%%%%%%%%%%%%%%%%%%%%%%%%%%%%%%%%%%%%%%%%%%%%%%%%%
%%% T(T)
\section{Transmission Dependency on Sample Temperature} \label{sec:analysis_fT_map}
Since I measured at low temperatures, between $50\,$mK and $4\,$K, I am also interested in changes of the transmission over the temperature. For this reason, I constantly logged the transmission and temperature of sample CPW3 during a warm-up process. The results can be seen in Figure \ref{fig:analysis_fT_map}.
\begin{figure}
    \centering
    \import{analysis/fT_map}{fT_map.pgf}
    \caption[Transmission behavior of sample CPW3 over temperature during warm-up]{Transmission behavior of sample CPW3 over temperature $T_\text{DUT}$ during warm-up process. In panel \textbf{\color{antiseeblau100}a} you can see the normalization transmission at $7\,$K. In panel \textbf{\color{antiseeblau100}b} the normalized transmission $T_{21}^\text{n}$ over the complete temperature range is shown. Four insets for close-ups are included as well.}
    \label{fig:analysis_fT_map}
\end{figure}

Since the thermometer of the sample holder ($T_\text{FMR}$) is not calibrated up to room temperature, in comparison to the magnet thermometer ($T_\text{magnet}$), the device under test (DUT) temperature $T_\text{DUT}$ was calculated by
\begin{align}
    T_\text{DUT}(t)=\left(\frac{1}{2}+\operatorname{atan}\left(\frac{t-t_0}{\Delta t}\right)\right)\cdot T_\text{FMR}(t)+\left(\frac{1}{2}-\operatorname{atan}\left(\frac{t-t_0}{\Delta t}\right)\right)\cdot T_\text{magnet}(t)\,.
\end{align}
The parameters $t_0$ and $\Delta t$ were chosen so that when the two measured temperatures begin to diverge at about $25-30\,$K, $T_\text{DUT}$ begins to converge from $T_\text{FMR}$ to $T_\text{magnet}$. This was visualized for a typical warm-up curve in Figure \ref{fig:analysis_fT_map_Tt}.
\begin{figure}
    \centering
    \import{analysis/fT_map}{T(t).pgf}
    \caption[Typical warm-up curve for the BlueFors setup]{Typical warm up curve over time of the BlueFors setup. \textbf{\color{seeblau100}$\boldsymbol{T}_\text{FMR}$} \& \textbf{\color[rgb]{0.571821,0.803166,0.878400}$\boldsymbol{T}_\text{magnet}$} are measured temperatures, whereas \textbf{\color{antiseeblau100}$\boldsymbol{T}_\text{DUT}$} is calculated by those temperatures and some $\operatorname{atan}$ weights. Panels \textbf{\color{antiseeblau100}a} \& \textbf{\color{antiseeblau100}b} are showing different time periods. The parameters are $t_0\approx4\,$h and $\Delta t\approx20\,$min} 
    \label{fig:analysis_fT_map_Tt}
\end{figure}

The most dominant feature in Figure \ref{fig:analysis_fT_map} is the smooth transition at about $75\,$K from a very homogeneous transmission distribution, given by normalization, to a rougher transmission distribution. This behavior can be explained very well by the collapse of superconductivity in the used high temperature superconducting microwave cables.

The vertical features at around $180\,$K are attributed to binning-algorithm errors and will not be discussed further.


Next, I would like to discuss features that need further investigation. For some frequencies, better transmission is obtained at higher temperatures, when the superconducting cables are in normal state. These features seem to be associated with a temperature dependence of the frequency. In particular, the frequency and temperature dependent features are magnified in the four insets of Figure \ref{fig:analysis_fT_map}. In itself, the FMR has no temperature dependent quantities, so these features must also be related to the transmission characteristics of the microwave cables.

An approach to explanation would be to assume an electro-magnetic resonator. In this case, the permittivity of the resonator would have to depend strongly on the temperature. Since the structure is in vacuum, only the cables and their terminations can form resonators, which depend on temperature and frequency.

At this point, further investigation is needed to determine the origin of these features. Also, a higher resolution measurement in the range from base temperature to about $10\,$K would be very interesting. In this temperature range, I expect a large number of interesting transmission features, especially in the order of magnitude of transmission variations due to FMR. Here I would like to emphasize the different orders of magnitude of transmissions variation caused by either temperature variations or typically measured FMR. Unfortunately, the measured warm-up was too fast for the slow frequency sweeps, which are optimized for a high dynamic range of transmission. Further, parameter optimization is needed for measurements in this low temperature range.

In early measurements on the HelioxVL setup I could see large differences in transmission compared to the FMR signal before and after helium transfer. During the transfer, the temperature in the dewar fluctuates quite a lot. Unfortunately, I did not log the temperature during these measurements and it remains questionable whether the presence of the transfer line might not also influence the applied magnetic field.