\chapter{Theory}
In this Chapter in order to understand the ferromagnetic resonance (FMR) technique I want to recall some basic concepts of magneto dynamics. Furthermore, I also want to give a short excursion into the coherence lengths of singlet and triplet Cooper pairs, since they are needed to define my samples dimensions reasonably. Finally, I want to present you the current state of the art, in the field of FMR measurements on thin ferromagnetic films.

%%%%%%%%%%%%%%%%%%%%%%%%%%%%%%%%%%%%%%%%%%%%%%%%%%%%%%%%%%%%%%%%%%%%%%%%%%%%%%%%%%%%%%%%%%%%%
\section{Landau-Lifshitz-Gilbert Model}
The dynamic magnetic properties of a ferromagnet can be described by damped precessional motion of the exchange-coupled magnetic moments.
If the externally applied magnetic fields and excitation energies are significantly smaller than the exchange coupling energy, we can describe the whole ferromagnet as a macroscopic magnetic spin by the Landau-Lifshitz-Gilbert (LLG) model. \cite{LANDAU199251, Gilbert2004}

The LLG differential equation for the magnetization $\vec{M}$ is given by
\begin{align}
\frac{\text{d}\vec{M}}{\text{d}t}=\underbrace{-\left|\gamma\right| \left(\vec{M}\times \mu_0 \vec{H}_\text{eff}\right)}_{(1)}
+\underbrace{\frac{\alpha}{\left|\vec{M}\right|}\left( \vec{M}\times \frac{\text{d}\vec{M}}{\text{d}t}\right)}_{(2)}\,,\label{formula:LLG}
\end{align}
where $\gamma$ is the gyro-magnetic ratio, $\alpha$ is the Gilbert damping parameter and $\mu_0$ is the vacuum permeability. The effective magnetic field is given by $\vec{H}_\text{eff}=\vec{H}+\vec{H}_\text{ani}$, with $\vec{H}$ being the externally applied magnetic field and $\vec{H}_\text{ani}$ the anisotropy field. 

For better understanding of the LLG model, we first neglect the Gilbert damping term (2) in equation \ref{formula:LLG}. With this neglection the LLG equation simply describes the precession of the magnetization $\vec{M}$ around the effective field $\vec{H}_\text{eff}$. Here the gyro-magnetic ratio is given by $\gamma_\text{e}=\frac{g_\text{e}\mu_\text{B}}{\hbar}=\frac{g_\text{e}q_\text{e}}{2m_\text{e}}$, where $\hbar$ is the reduced Planck constant, $\mu_\text{B}$ the Bohr magneton, $g_\text{e}$, $q_\text{e}$ and $m_\text{e}$ describe the $g$-factor, charge and mass of the free electron respectively. Since the magnetism in ferromagnets, especially in Co, is caused by the electron spins, we can assume a $g$-factor $g\approx 2$ and thus a gyromgnetic ratio of $\frac{\gamma}{2\pi}=28.025$\,GHzT$^{-1}$.
The precession frequency $\omega_\text{res}$ is given by
\begin{align}
    \omega_\text{res}=\gamma\mu_0|\vec{H}_\text{eff}|\,,
\end{align}
resulting in typical resonance frequencies $\omega_\text{res}/2\pi$ in the GHz range, for a few Tesla of applied external magnetic field.

Now we consider the phenomenological Gilbert term (2) from equation \ref{formula:LLG}. After the displacement of $\vec{M}$, it precesses in a spiral trajectory back to its equilibrium position. This relaxation is due to the scattering of phonons and magnons and can be characterised by the relaxation rate $\kappa$. This rate is related to the frequency linewidth by $\Delta\omega=\frac{1}{2\kappa}$, where the linewidth $\Delta\omega$ is usually defined as full width at half maximum (FWHM) and $\kappa$ as the inverse of the half width at half maximum (HWHM).
Now the linewidth $\Delta\omega$ is  connected to the Gilbert parameter $\alpha$ by
\begin{align}
    \Delta\omega=2\alpha\omega_\text{res}+\Delta\omega_0\,.
\end{align}
The inhomogeneous broadening $\Delta\omega_0$ is caused by magnetic inhomogeneities or surface effects. \cite{Beaujour2009}

In order to solve the LLG differential equation \ref{formula:LLG} we will neglect for now the field anisotropy $\vec{H}_\text{ani}=0$. To this end we split the applied magnetic field $\vec{H}$ and magnetization $\vec{M}$ into static ($\vec{H}_0, \vec{M}_0$) and dynamic ($\vec{h}, \vec{m}$) components.
\begin{align}
    \vec{H}&=\vec{H}_0+\vec{h}(t)\label{formula:LLG_field}\\
    \vec{M}&=\vec{M}_0+\vec{m}(t)
\end{align}
Next we assume a static magnetic field in $z$-direction and a dynamic field in $x$- and $y$-direction $\vec{H}=\left(h_x(t),h_y(t),H_0\right)$. If the dynamic magnetic field is much smaller than the static magnetic field, we can also write for the magnetization $\vec{M}=\left(m_x(t),m_y(t),M_0\right)$. Here is $M_0$ the absolute static magnetization. The solution is given by $\vec{m}=\boldsymbol{\chi}\vec{h}$. The two-dimensional Polder tensor $\boldsymbol{\chi}$ is given by
\begin{align}
    \boldsymbol{\chi}=
    \left(
    \begin{array}{cc}
        \chi_{11} & i\chi_{12} \\
        -i\chi_{12} & \chi_{22}
    \end{array}
    \right)\,.
\end{align}
\cite{gurevich1996magnetization}

Since we neglect magnetic field anisotropy, the diagonal elements are the same $\chi=\chi_{11}=\chi_{22}$. Further, we neglect every higher damping therm $\mathcal{O}(\alpha^2)$, in order to get the linear response of a ferromagnet in an external field. Finally, the Polder susceptibility is then given by
\begin{align}
    \chi(\omega,H_0)=\frac{\omega_M\left(\gamma\mu_0H_0-i\Delta\omega\right)}
    {\left( \omega_\text{res}(H_0)\right)^2-\omega^2-i\omega\Delta\omega}\,. \label{eq:chi}
\end{align}
Here is the magnetization frequency  $\omega_M=\gamma\mu_0 M_0$ and the resonance frequency $\omega_\text{res}$. In Figure \ref{fig:theo_chi}, you can see the qualitative behaviour of the Polder susceptibility $\chi(\omega)$ around the resonance frequency $\omega_\text{res}$. \cite{schneider2007, Kalarickal2006}
\begin{figure}
     \centering
     \begin{subfigure}[b]{.33\textwidth}
         \centering
    \import{theory/chi}{chi.pgf}
         \caption{Complex susceptibility $\chi(\omega)$}
         \label{fig:theo_chi}
     \end{subfigure}
     \hfill
     \begin{subfigure}[b]{.6\textwidth}
         \centering
    \import{theory/chi}{wres.pgf}
         \caption{Resonance frequency $\omega_\text{res}/2\pi(\mu_0H_0)$}
         \label{fig:theo_wres}
     \end{subfigure}
        \caption[Schematic of the Polder susceptibility and simulation of the resonance frequency]{\textbf{(a)} Schematic of the complex Polder susceptibility $\chi(\omega)$ in arbitrary units, according to equation \ref{eq:chi}. The maximum of the real part (\textbf{\color{seeblau100}blue}) and the zero crossing of the imaginary part (\textbf{\color{antiseeblau100}magenta}) are marking the resonance frequency $\omega_\text{res}$. The FWHM of the real part and the distance between the minimum and maximum of the imaginary part is marking the line width $\color[rgb]{0.520000,0.520000,0.520000}\pmb{\Delta}\boldsymbol{\omega}$.
        \\\textbf{(b)} Simulation of the ferromagnetic resonance frequency $\omega_\text{res}/2\pi(\mu_0H_0)$, according to equation \ref{formula:kittel1}-\ref{formula:kittel3}. Shown are the curves for the three most common shape anisotropies, that of a sphere (\textbf{\color[rgb]{0.520000,0.520000,0.520000}grey}), an in-plane (\textbf{\color{seeblau100}blue}) or an out-of-plane (\textbf{\color{antiseeblau100}magenta}) magnetized thin film. Parameters: $M_0=8\,$kOe \& $\gamma=28.025\,$GHzT$^{-1}$}
        \label{fig:theo_chi_wres}
\end{figure}

%%%%%%%%%%%%%%%%%%%%%%%%%%%%%%%%%%%%%%%%%%%%%%%%%%%%%%%%%%%%%%%%%%%%%%%%%%%%%%%%%%%%%%%%%%%%%
\newpage
\section{Kittel Formula}\label{sec:theo_Kittel}
Most macroscopic ferromagnets have two opposite favorable directions in which they are most easily magnetized. The so-called easy-axis, which is parallel to the two directions, can be of different origin. In the following simplest case the sample shape anisotropy is treated. Here the magnetic field $\vec{H}$ can be written as follows:
\begin{align}
    \vec{H}=\vec{H}_0+\vec{H}_\text{demag}+\vec{h}(t)\,.
\end{align}

The demagnetization field is given by $\vec{H}_\text{demag}=\vec{N}\cdot\vec{M}$, where the spatially independent demagnetization tensor $\vec{N}$ is in diagonal form, with elements $N_{x,y,z}$.
The resonance frequency can be written as
\begin{align}
    \omega_\text{res}=\gamma\mu_0\sqrt{\left(H_0+(N_y-N_z)M_0\right)\left(H_0+(N_x-N_z)M_0\right)}\,,\label{formula:kittel}
\end{align}
if the applied magnetic field is in the direction of the $z$-axis.

These demagnetisation factors $N_{x,y,z}$ are strongly dependent on the sample geometry. The three most common geometries are discussed below. 
\begin{enumerate}
    \item spherical geometry ($N_{x,y,z}=1/3$).
    \begin{align}
        \omega_\text{res}^\circ=\gamma\mu_0H_0\label{formula:kittel1}
    \end{align}
    \item thin out-of-plane magnetized film ($N_{x,y}=0, N_z=1$).
    \begin{align}
        \omega_\text{res}^\perp=\gamma\mu_0(H_0-M_0)\label{formula:kittel2}
    \end{align}
    \item thin in-plane magnetized film ($N_{x,z}=0, N_y=1$).
    \begin{align}
        \omega_\text{res}^\parallel=\gamma\mu_0\sqrt{H_0(H_0+M_0)}\label{formula:kittel3}
    \end{align}
\end{enumerate}
The ferromagnetic resonance frequencies are simulated in Figure \ref{fig:theo_wres}. 

If there are no other anisotropies than shape anisotropy, $M_0$ is replaced by the saturation magnetization $M_\text{s}$. \cite{kittel1996}

If uniaxial field anisotropy is present, $H_0$ is replaced by $H_\text{eff}=H_0+H_\text{ani}$. For thin in-plane magnetized films, the uniaxial field anisotropy is typically on the order of a few mT. \cite{Kalarickal2006}

At this point, it should be noted that we do not know with certainty whether the easy-axis in thin Co films is in-plane, because out-of-plane easy-axis is also possible. With a suitable substrate, such as Au or Pt, single-digit monolayers of Co, and a suitable cap, such as Au or Ag, out-of-plane magnetization of Co may be present. In addition, there are studies of Co films as thin as $40\,$nm that both at normal and oblique incidence of atomic current during electron beam evaporation Co always exhibit in-plane easy-axis.
At this point, I suspect that Co exhibits robust in-plane magnetization and only shows out-of-plane magnetization under very special conditions. \cite{Su2003, McGee1993, Ujfalussy1996, SZUNYOGH1997}

Furthermore, it should be considered that superconductors, such as aluminum, have a critical magnetic field of only a few mT. Only for very thin films, in the single-digit nanometer range, aluminum can achieve an in-plane critical field in the Tesla range. To use a finite ferromagnetic resonant frequency, within the critical magnetic field, an in-plane geometry must be used. \cite{Caplan1965,Tedrow1982}

\newpage
%%%%%%%%%%%%%%%%%%%%%%%%%%%%%%%%%%%%%%%%%%%%%%%%%%%%%%%%%%%%%%%%%%%%%%%%%%%%%%%%%%%%%%%%%%%%%
\section{Co-planar Waveguide}
In order to generate an alternating magnetic field distribution $\vec{h}$ I use a co-planar wave guide (CPW). The CPW consists of an inner conductor, with width $w$, an infinitesimal height and two neighboring ground pads. For further descriptions I use the laboratory reference frame, shown in Figure \ref{fig:CPW_schematic}, with $x$ perpendicular to the CPW plane, $z$ in direction of the inner conductor and $y$ being perpendicular to $x$ and $z$.
\begin{figure}
    \centering
    %\vspace{-2.5mm}
    \import{theory/CPW}{CPW.pdf_tex}
    %\vspace{-2.5mm}
    \caption[Cross-section schematic of a CPW]{Cross-section schematic of a co-planar wave guide. The laboratory reference frame is centered on the inner conductor with width $w$ and infinitesimal height. The magnetic field $\color{seeblau100}\vec{h}$ curles around the current $\color{seeblau100}\vec{I}$ right-handed wise. Whereas the dark grey areas are conducting, the light grey area is insulating. The conducting area beside and below the inner conductor are separated by an insulating layer and connected to ground.}
    \label{fig:CPW_schematic}
\end{figure}

For an infinitely thin sheet the field distribution $h_0$ at the position $(x,y)=(0,0)$, the middle of the inner conductor, can be related to the current density $j$ or the current $I$, by
\begin{align}
    h_0=\frac{j}{2}=\frac{I}{2w}\,.
\end{align}
We also assume that the ground pads are infinitely large and so the current density through them is converging to zero.

Furthermore, we are interested in the field distribution $\vec{h}(x,y)$ in the $x$-$y$-plane. In vacuum, meaning with relative magnetic permittivity $\mu=1$, we can approximate this situation with the Karlqvist equations \cite{karlqvist1954calculation}, given by
\begin{align}
    h_y(x,y)&=\frac{h_0}{\pi} \left[ \operatorname{atan}\left(\frac{y+w/2}{x}\right) - \operatorname{atan}\left(\frac{y-w/2}{x} \right) \right] \label{CPW1}\\
    h_x(x,y)&=\frac{h_0}{2\pi} \ln \left[ \frac{\left(y+w/2 \right)^2+x^2}{\left(y-w/2 \right)^2+x^2}\right]\,. \label{CPW2}
\end{align}
These equations neglect the presence of the ground pads and a solution is visualized in Figure \ref{fig:CPW_field}.
\begin{figure}
    \centering
    \import{theory/field}{field.pgf}
    \caption[Magnetic field distribution of a CPW]{Magnetic field distribution $\vec{h}(x,y)$ of a CPW according to Karlqvist equations \ref{CPW1} \& \ref{CPW2}. In \textbf{\color{seeblau100} blue} you can see the infinitely thin inner conductor of a CPW, with width $w=300$\,\textmu m. However, the ground pads are neglected. The \textbf{\color{antiseeblau100} magenta} arrows are indicating the direction, whereas their length and the color scale are indicating the absolute magnetic field $|\vec{h}(x,y)|/h_0$\,.}
    \label{fig:CPW_field}
\end{figure}

For good electrical transmission through the CPW, its impedance must match to the impedance of the cabling. The impedance of the CPW can be adjusted by the width of the CPW's inner conductor and its distance to the ground pad. For more information see textbooks on high frequency technology.  \cite{wadell1991transmission, wiley2001CPW}

%%%%%%%%%%%%%%%%%%%%%%%%%%%%%%%%%%%%%%%%%%%%%%%%%%%%%%%%%%%%%%%%%%%%%%%%%%%%%%%%%%%%%%%%%%%%%
\section{Coherence Lengths} \label{sec:coherence_length}
The supercurrent through a superconductor/ferromagnet/superconductor (S/F/S) contact depends essentially on the thickness of the ferromagnet. However, this thickness affects the magnetic properties, which in turn determine the FMR. In general, we want to design an S/F/S contact, so we can measure a difference in electronic transport for short-range singlet and long-range triplet superconductivity. To this end, we consider the corresponding decay lengths in the ferromagnet.

For singlet superconductivity we are interested in the superconducting correlation decay length in a ferromagnet $\xi_\text{f}$. For this purpose, we assume diffusive transport, i.e., the mean free path $\ell$ of the electron scattering is small compared to the coherence length. This is the case if there are many defects in the lattice, also called the dirty limit. The diffusion constant is given by $D_\text{f}=\frac{1}{3}v_\text{F} \ell$, where $v_\text{F}$ is the Fermi velocity. Considering the exchange field $h_\text{ex}$, we can write
\begin{align}
    \xi_\text{f}=\sqrt{\frac{D_\text{f}}{h_\text{ex}}}=\sqrt{\frac{\hbar v_\text{F} \ell}{3h_\text{ex}}}\,.
\end{align}
For Co the electron-scattering mean free path is given by $\ell_\text{Co}=7.8\,$nm, whereas the exchange field is given by $h_\text{ex, Co}=5.15\cdot10^{-21}\,$J. This results in a decay length of $\xi_\text{f, Co}=3.7$\,nm. \cite{buzdin2006,gall2016,aharoni2000}
% https://www.wolframalpha.com/input/?i=sqrt%28%28reduced+planck+constant*2.55*10%5E5+m+%2F+s*7.77+nm%29%2F%283*5.15*10%5E%28-21%29J%29%29

The penetration depth of long-range triplet pairs $\xi_\varepsilon=\sqrt{\frac{D_\text{f}}{\varepsilon}}$ into the ferromagnet, is of the same order as the penetration depth of singlet pairs into a normal conductor $\xi_\varepsilon\approx\xi_\text{n}$. In practice, we replace the energy term $\varepsilon$ by the thermal energy $k_\text{B}T$ and multiply a factor of $2\pi$. This leads to the approximation
\begin{align}
    \xi_\text{n}=\sqrt{\frac{\hbar v_\text{F}\ell}{6\pi k_\mathrm{B}T}}\,.
\end{align}
Again, the mean free path of electron scattering for Co is $\ell_\text{Co}=7.8\,$nm and we assume a temperature of $T=300\,$mK. This temperature only has to be smaller than the critical temperature of the superconductor used, aluminum $T_\text{c, Al}=1,2$\,K and is reasonably achievable with the used setup. Taking this into account, the decay length for long-range triplet pairs is $\xi_{\text{n, Co}}(T=300\,\text{mK})= 52$\,nm. \cite{Ujfalussy1996,gall2016,bergeret2001, buckel2013}
% https://www.wolframalpha.com/input/?i=sqrt%28%28%5Chbar*2.55*10%5E5+m+%2F+s*7.77+nm%29%2F%283*2%5Cpi*1.381*10%5E%7B-23%7D+J%2FK*0.3K%29%29

In summary, the thickness of the Co should now be somewhere between $3.7$ and $52\,$nm to measure a significant difference in electronic transport between singlet and triplet superconductivity. However, a lower Co thickness of, say, $3\,$nm, can also be advantageous. On the one hand, singlet superconductivity can be studied very well, on the other hand, the effect of triplet superconductivity will still be easy to observe.

%%%%%%%%%%%%%%%%%%%%%%%%%%%%%%%%%%%%%%%%%%%%%%%%%%%%%%%%%%%%%%%%%%%%%%%%%%%%%%%%%%%%%%%%%%%%%
\section{State of the Art of Research}
In this Section I will focus on the presentation of publications related to broadband (bb) ferromagnetic resonance (FMR) measurements on thin magnetic films. As a quick reminder I use a $2$-port vector network analyzer (VNA) and couple my thin Co films with a co-planar waveguide (CPW) at cryogenic temperatures ($0.1$ to $4\,$K) and in-plane magnetic fields.

Kalarickal et al. present various relative broadband FMR methods and the resulting determination of the Gilbert damping with it of thin ($50$ to $100\,$nm) in-plane permalloy layers. It is found that the methods strip-line FMR, VNA FMR and pulsed inductive microwave magnetometer give comparable results. \cite{Kalarickal2006}

Maier-Flaig et al. also deal with the characterization of thin permalloy films. The CPW VNA technique is also used, but the magnetic field is applied out-of-plane. This allows the clever application of numerical methods for background treatment. Thus, uncalibrated measurements can be made and noise can be removed afterwards. \cite{maierflaig2018,maierflaig2017}

Tamaru et al. deal with the signal enhancement of VNA FMR by additionally modulating the in-plane applied magnetic field in the Hertz frequency range. A CPW and a stack of FeB($1.6\,$nm)/ W($0.1\,$nm)/FeB($1.1\,$nm) were used for this purpose. \cite{Tamaru2018}
%https://aip.scitation.org/doi/pdf/10.1063/1.5022762

%González et al are engaged in the development of a magnetoimpedance sensor. Thin permalloy structures and the strip-line VNA technique are used. The strip-line technique is the earliest form of planar transmission line, from which the CPW has evolved.
%https://www.sciencedirect.com/science/article/abs/pii/S0263224118304603

Harward et al. have developed a bbFMR system with impressive $10\,$MHz to $70\,$GHz frequency bandwidth. Furthermore, they can vary the sample temperature from $27$ to $350\,$K. They claim to achieve a sensitivity of less than one monolayer of crystalline iron. For their measurement they use a CPW and can apply the magnetic field simultaneously in- and out-of-plane. \cite{Harward2011}
%https://aip.scitation.org/doi/pdf/10.1063/1.3641319

Denysenkov and Grishin report a strip-line bbFMR setup in a cryostat ($4$ to $420\,$K). This allowed them to resolve the signal of YIG films with $1\,$\textmu m thickness. The magnetic field can be applied in-plane or out-of-plane through a rotatable sample holder. \cite{Denysenkov2003}
%https://aip.scitation.org/doi/pdf/10.1063/1.1581395

Bilzer et al have studied the use of all four possible scattering parameters of a $2$-port VNA. They were able to determine the resonance frequency with only the forward transmission scattering parameter with less than one percent deviation compared to all four scattering parameters used. A CPW and a $40\,$nm thick in-plane permalloy film were used. \cite{Bilzer2007}
%https://aip.scitation.org/doi/pdf/10.1063/1.2716995

In summary, permalloy was mostly used for good signal-to-noise ratio. It was found that the CPW VNA bbFMR method gives equivalent results to other bbFMR methods. The number of scattering parameters used is not important. It is possible to implement these methods for in- or out-of-plane magnetic fields and at cryogenic temperatures.
