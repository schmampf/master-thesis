% Introduction
% Previous or relevant references
% The goal of the project
% How that goal was met
% Key results
% What makes your results unique or noteworthy

% Through ferromagnetic resonance excitation, a long-range spin triplet supercurrent can flow through a ferromagnetic barrier, according to the theoretical prediction of Hikino et al [1, 2]. For electronic transport measurements through a superconductor-ferromagnet-superconductor contact, the dimension, such as the thickness, of the ferromagnet is crucial. Therefore, Co films as thin as 3 nm were prepared by electron beam evaporation with a co-planar waveguide on top. Transmission measurements were performed at cryogenic temperatures using an in-plane magnet and a broadband vector network analyzer.
%A quantitative correlation between ferromagnetic resonance frequency and applied magnetic field could be established.

% Entsprechend der Theorie nach Hikino et al \cite{Hikino2007,Hikino_2011} kann durch ferromagnetische Resonanzanregung ein langreichweitiger Spin-Triplett-Suprastrom durch eine ferromagnetische Barriere flie\ss en. Für elektronische Transportmessungen durch einen Supraleiter-Ferromagnet-Supraleiter-Kontakt sind Abmessungen wie die Dicke des Ferromagneten entscheidend. Daher wurden Kobaltfilme mit einer Dicke von $3\,$nm durch Elektronenstrahlverdampfung mit einem koplanaren Wellenleiter auf der Oberseite hergestellt. Transmissionsmessungen wurden bei kryogenen Temperaturen mit einem parallelen Magnetfeld und einem breitbandigen Vektor-Netzwerkanalysator durchgeführt.
% Es konnte eine quantitative Korrelation zwischen der ferromagnetischen Resonanzfrequenz und dem angelegten Magnetfeld hergestellt werden.


%\noindent\begin{minipage}[b][.49\textheight]{\textwidth}
\chapter*{Zusammenfassung}
Es ist bekannt, dass durch ferromagnetische Resonanz (FMR) Anregungen ein reiner Spinstrom aus einem präzidierenden Ferromagneten (F) in ein normales Metall (N) eingekoppelt werden kann. Dieser Spinstrom kann in spintronischen Bauelementen zur Kodierung und Verarbeitung digitaler Informationen genutzt werden.

Ähnlich wie im Fall von F/N-Systemen wurde kürzlich vorgeschlagen, dass ein vollständig spinpolarisierter (Spin-Triplett) supraleitenden Strom (Suprastrom) auch in Supraleiter/Ferromagneten (S/F) Systemen erzeugt werden kann und somit zur Nutzung von Spintronik-Operationen mit geringen Energieverlusten im supraleitenden Regime eingesetzt werden kann. Im Gegensatz zu konventionellen Supraströmen, die aus antiparallel ausgerichteten (Spin-Singulett) Cooper-Paaren von Elektronen bestehen, sind Spin-Triplett Supraströme nicht nur spinpolarisiert, sondern auch langreichweitig innerhalb eines F. Folglich könnte man durch Realisierung eines S/F/S-Josephson-Bauelements und Anregung einer FMR der F-Schicht von einem Zustand, in dem kein Transport zwischen den beiden S-Schichten stattfindet (aufgrund des schnellen Abklingens des Spin-Singlet Suprastroms in F), zu einem Zustand wechseln, in dem die beiden S-Schichten stattdessen über einen Spin-Triplett Suprastrom gekoppelt sind, der durch die FMR-Anregung von F ausgelöst wird. Um solche supraleitenden Bauelemente zu realisieren, die über FMR-Anregung geschaltet werden können, scheint es entscheidend zu sein, das FMR-Signal von ultradünnen F-Schichten auslösen und detektieren zu können.

Basierend auf diesen Beweggründen wird in dieser Arbeit ein Aufbau zur hochempfindlichen Messung des FMR-Signals von ultradünnen Co (F)-Schichten (bis zu $\approx3\,$nm Dicke) diskutiert. Die ultradünnen Co-Schichten werden durch Elektronenstrahlverdampfung erzeugt, gefolgt von der Herstellung eines koplanaren Wellenleiters darauf. Mikrowellentransmissionsmessungen werden bei niedrigen Temperaturen unter Verwendung eines in der Ebene liegenden Magnetfeldes und eines breitbandigen Vektornetzwerkanalysators durchgeführt. Eine quantitative Beziehung zwischen der FMR Frequenz und dem angelegten Magnetfeld wird hergestellt. Die Funktionsweise des Aufbaus einschlie\ss lich seiner Optimierung in Bezug auf die Rauschunterdrückung des FMR-Signals wird diskutiert. Die erzielten Ergebnisse zeigen die Eignung des Aufbaus für die Charakterisierung der Transporteigenschaften von S/F/S-Bauteilen unter FMR-Anregung.

%\end{minipage}
%https://www.nature.com/articles/s41563-018-0058-9.pdf

%\noindent\begin{minipage}[.49\textheight]{\textwidth}
\chapter*{Abstract}
It is well-established, that through ferromagnetic resonance (FMR) excitation a pure spin current can be injected from a precessing ferromagnet (F) into a normal metal (N) material. This spin current can be used in spintronic devices for the encoding and processing of digital information.

Similar to the case of F/N systems, very recently it has been suggested that a fully spin-polarized (spin-triplet) superconducting current (supercurrent) can be generated also in superconductor/ferromagnet (S/F) systems, and therefore applied to perform spintronics operations with low energy dissipation in the superconducting regime.	Unlike conventional supercurrents that consist of antiparallel-aligned (spin-singlet) Cooper pairs of electrons, spin-triplet supercurrents are not only spin-polarized but also long-ranged inside a F. As a result, by realizing a S/F/S Josephson device and exciting a FMR of the F layer, one could switch from a state where no transport occurs between the two S layers (due to the quick decay of the spin-singlet supercurrent in F) to a state where the two S layers are instead coupled via a spin-triplet supercurrent triggered via the FMR excitation of F. To realize such superconducting devices, which can be switched via FMR excitation, it appears crucial to be able to trigger and detect the FMR signal from ultrathin F layers.

Based on these motivations, in this thesis a setup for high-sensitivity measurement of the FMR signal from ultrathin Co (F) thin films (down to $\approx3\,$nm in thickness) is discussed. The ultrathin Co thin films are grown by electron beam evaporation followed by the fabrication of a co-planar waveguide on top of them. Microwave transmission measurements are performed at low temperatures, using an in-plane magnetic field and a broadband vector network analyzer. A quantitative relation between FMR frequency and applied magnetic field is derived. The performance of the setup including its optimization in terms of noise reduction of the FMR signal is discussed. The results obtained demonstrate the suitability of the setup for the characterization of the transport properties of S/F/S devices under FMR excitation.
%\end{minipage}

